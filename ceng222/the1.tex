\documentclass[12pt]{article}
\usepackage[utf8]{inputenc}
\usepackage{float}
\usepackage{amsmath}


\usepackage[hmargin=3cm,vmargin=6.0cm]{geometry}
%\topmargin=0cm
\topmargin=-2cm
\addtolength{\textheight}{6.5cm}
\addtolength{\textwidth}{2.0cm}
%\setlength{\leftmargin}{-5cm}
\setlength{\oddsidemargin}{0.0cm}
\setlength{\evensidemargin}{0.0cm}

\newcommand{\HRule}{\rule{\linewidth}{1mm}}

%misc libraries goes here
\usepackage{tikz}
\usetikzlibrary{automata,positioning}

\begin{document}

\noindent
\HRule \\[3mm]
\begin{flushright}

                                         \LARGE \textbf{CENG 222}  \\[4mm]
                                         \Large Statistical Methods for Computer Engineering \\[4mm]
                                        \normalsize      Spring '2017-2018 \\
                                           \Large   Take Home Exam 1 \\
                    \normalsize Deadline: May 25, 23:59 \\
                    \normalsize Submission: via COW
\end{flushright}
\HRule

\section*{Student Information }
%Write your full name and id number between the colon and newline
%Put one empty space character after colon and before newline
Full Name : Beyazıt Yalçınkaya \\
Id Number : 2172138 \\

% Write your answers below the section tags
\section*{Answer 3.8}

When the user tries to find her password among the $4$ possible alternatives, it is equally likely that she finds the right one at the first, second, third or fourth trial. Thus, the pmf of $X = 0$, $X = 1$, $X = 2$, and $X = 3$ are all equal as follows,
\begin{equation}
	P(0) = P(1) = P(2) = P(3) = 0.25
\end{equation}
\begin{equation}
	E(X) = \sum_{x = 0}^{3} xP(x) = \big(0 + 1 + 2 + 3\big)(0.25) = 1.5
\end{equation}
\begin{equation}
	Var(X) = \sum_{x = 0}^{3} (x - 1.5)^2P(x) = \big((-1.5)^2 + (-0.5)^2 + (0.5)^2 + (1.5)^2\big)(0.25) = 1.25
\end{equation}

\section*{Answer 3.15}

\subsection*{(a)} Probability of at least one hardware failure can be computed as follows,
\begin{equation}
	1 - P_{(X, Y)}(0, 0) = 1 - 0.52 = 0.48
\end{equation}

\subsection*{(b)} It is known that $X$ and $Y$ are independent if $P_{(X, Y)}(x, y) = P_{X}(x)P_{Y}(y)$ for all values of $x$ and $y$.
By the Addition Rule we can find $P_X(0)$ and $P_Y(0)$ as follows,
\begin{equation}
\begin{split}
	P_X(0) = P\{X = 0\} = \sum_{y = 0}^{2}P_{(X, Y)}(0, y) = 0.52 + 0.14 + 	0.06 = 0.72 \\
	P_Y(0) = P\{Y = 0\} = \sum_{x = 0}^{2}P_{(X, Y)}(x, 0) = 0.52 + 0.20 + 	0.04 = 0.76
\end{split}
\end{equation}
Now, let's test the statement $P_{(X, Y)}(x, y) = P_{X}(x)P_{Y}(y)$ for $x = 0$ and $y = 0$,
\begin{equation}
\begin{split}
	P_{(X, Y)}(x, y) & = P_{X}(x)P_{Y}(y) \\
	P_{(X, Y)}(0, 0) & = P_{X}(0)P_{Y}(0) \\
	0.52 & \neq 0.5472
\end{split}
\end{equation}
Hence, $X$ and $Y$ are not independent, it follows that $X$ and $Y$ are dependent.

\section*{Answer 3.19}

Before starting further analysis on the given cases, we need to calculate $E(X)$, $Var(X)$, $E(Y)$, and $Var(Y)$ as follows,
\begin{equation}
\begin{split}
	E(X) & = (2)(0.5) + (-2)(0.5) = 0 \\
	Var(X) & = (2)^2(0.5) + (-2)^2(0.5) = 4 \\
	E(Y) & = (4)(0.2) + (-1)(0.8) = 0 \\
	Var(Y) & = (4)^2(0.2) + (-1)^2(0.8) = 4
\end{split}
\end{equation}
Expected value and variance of the total profit for strategies are computed below. \\
(a) Buying 100 share of A means collecting a profit of $A = 100X$. $E(A)$ and $Var(A)$ can be computed as follows,
\begin{equation}
\begin{split}
	E(A) & = (100)E(X) = (100)(0) = 0 \\
	Var(A) & = (100)^2Var(X) = (100)^2(4) = 40000
\end{split}
\end{equation}
(b) Buying 100 share of B means collecting a profit of $B = 100Y$. $E(B)$ and $Var(B)$ can be computed as follows,
\begin{equation}
\begin{split}
	E(B) & = (100)E(Y) = (100)(0) = 0 \\
	Var(B) & = (100)^2Var(Y) = (100)^2(4) = 40000
\end{split}
\end{equation}
(c) Buying 50 share of A and 50 share of B means collecting a profit of $C = 50X + 50Y$. $E(C)$ and $Var(C)$ can be computed as follows,
\begin{equation}
\begin{split}
	E(C) & = (50)E(X) + (50)E(Y) = (50)(0) + (50)(0) = 0 \\
	Var(C) & = (50)^2Var(X) + (50)^2Var(Y) = (50)^2(4) + (50)^2(4) = 20000
\end{split}
\end{equation}

\section*{Answer 3.29}

In this problem, we will use Poisson distribution and Bayes Rule for two events. $P\{H\}$ denotes probability of high risk driver and $P\{L\}$ denotes probability of low risk driver. In the problem it is given that $P\{H\} = 0.2$, $P\{L\} = 0.8$. Using Poisson distribution, we can find $P\{0 | H\}$ and $P\{0 | L\}$. Note that, for $P\{0 | H\}$, $\lambda_H = 1$ and $x_H = 0$ and for $P\{0 | L\}$, $\lambda_L = 0.1$ and $x_L = 0$.
\begin{equation}
\begin{split}
	P\{0 | H\} & = e^{-\lambda_H} \frac{\lambda_H^{x_H}}{x_H!} = e^{-1} \frac{1^0}{0!} = e^{-1} \\
	P\{0 | L\} & = e^{-\lambda_L} \frac{\lambda_L^{x_L}}{x_L!} = e^{-0.1} \frac{0.1^0}{0!} = e^{-0.1}
\end{split}
\end{equation}
After finding all necessary values, using Bayes Rule for two events, we can find $P\{H | 0\}$ as follows,
\begin{equation}
\begin{split}
	P\{H | 0\} & = \frac{P\{0 | H\}P\{H\}}{P\{0 | H\}P\{H\} + P\{0 | L\}P\{L\}} \\
	P\{H | 0\} & = \frac{(e^{-1})(0.2)}{(e^{-1})(0.2) + (e^{-0.1})(0.8)} \\
	P\{H | 0\} & = 0.0923
\end{split}
\end{equation}

\end{document}

