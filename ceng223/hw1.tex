\documentclass[12pt]{article}
\usepackage[utf8]{inputenc}
\usepackage{float}
\usepackage{amsmath}


\usepackage[hmargin=3cm,vmargin=6.0cm]{geometry}
%\topmargin=0cm
\topmargin=-2cm
\addtolength{\textheight}{6.5cm}
\addtolength{\textwidth}{2.0cm}
%\setlength{\leftmargin}{-5cm}
\setlength{\oddsidemargin}{0.0cm}
\setlength{\evensidemargin}{0.0cm}

%misc libraries goes here
%\usepackage{fitch}


\begin{document}

\section*{Student Information } 
%Write your full name and id number between the colon and newline
%Put one empty space character after colon and before newline
Full Name :  Beyazıt Yalçınkaya\\
Id Number :  2172138\\

% Write your answers below the section tags
\section*{Answer 1}
\vspace{10px}

\begin{table}[H]
\centering
\caption{Answer 1.1}
\label{my-label1}
\begin{tabular}{|c|c|c|}
\hline
$p$&$q$&$(\neg q \land(p \rightarrow q)) \rightarrow \neg p$\\ \hline
$T$&$T$&$T$\\ \hline
$T$&$F$&$T$\\ \hline
$F$&$T$&$T$\\ \hline
$F$&$F$&$T$\\ \hline
\end{tabular}
\end{table}  

\vspace{10px}

\begin{table}[H]
\centering
\caption{Answer 1.2}
\label{my-label2}
\begin{tabular}{|c|c|c|c|}
\hline
$p$&$q$&$r$&$((p \lor q) \land (\neg p \lor r)) \rightarrow (q \lor r)$\\ \hline
$T$&$T$&$T$&$T$\\ \hline
$T$&$F$&$T$&$T$\\ \hline
$F$&$T$&$T$&$T$\\ \hline
$F$&$F$&$T$&$T$\\ \hline
$T$&$T$&$F$&$T$\\ \hline
$T$&$F$&$F$&$T$\\ \hline
$F$&$T$&$F$&$T$\\ \hline
$F$&$F$&$F$&$T$\\ \hline
\end{tabular}
\end{table} 



\section*{Answer 2}

\vspace{10px}

\begin{equation} 
\label{eq1}
\begin{split}
(p \rightarrow q) \lor (p \rightarrow r) & \equiv p \rightarrow (q \lor r) \qquad \textit{Using Table 7} \\
& \equiv \neg(q \lor r) \rightarrow \neg p \qquad \textit{Using Table 7} \\
& \equiv (\neg q \land \neg r) \rightarrow \neg p \qquad \textit{De Morgan's Laws} \\
\end{split}
\end{equation}

We started from the first equation and we managed to get the second equation by using some logical equivalences from given tables. Thus, it can be concluded that $(p \rightarrow q) \lor (p \rightarrow r)$ and $(\neg q \land \neg r) \rightarrow \neg p$ are logically equivalent.



\section*{Answer 3}

\vspace{10px}

\begin{enumerate}
	\item (a) $Every \ cat \ has \ at \ least \ one \ dog \ friend.$  \\ (b) $Some \ cats \ are \ friends \ with \ all \ dogs.$
	\item
	(a) $\neg (\exists x \exists y (\neg Customer(x) \land Meal(y) \land Eats(x,\ y)))$ \\
	(b) $\exists x \exists y (Chef(x) \land Meal(y) \land \neg Cooks(x,\ y))$ \\
	(c) $\exists x (Customer(x) \land \exists y(Chef(y) \land \neg (\exists z (Meal(z) \land Cooks(y,\ z) \land \neg Eats(x,\ z)))))$ \\
	(d) $\forall x (Chef(x) \rightarrow \exists y (Chef(y) \land Knows(x,\ y) \land \neg (\exists z (Meal(z) \land \neg Cooks(x,\ z) \land \neg Cooks(y,\ z)))))$
\end{enumerate}







\section*{Answer 4}

\vspace{5px}

\quad The truth value of $\neg p$ is given as $T$, so it can be concluded that the truth value of $p$ is $F$, since $\neg p \equiv \neg F$ and $\neg F \equiv T$. Thus, the following truth table can be constructed.

\vspace{5px}
\begin{table}[H]
\centering
\caption{Answer 4}
\label{my-label3}
\begin{tabular}{|c|c|c|}
\hline
$p$&$q$&$ p \rightarrow q$\\ \hline
$F$&$T$&$T$\\ \hline
$F$&$F$&$T$\\ \hline
\end{tabular}
\end{table}

From the table it can be seen that, the truth value of $\neg q$ cannot be determined as $T$. Since the compound proposition is a tautology, the truth value of the compound proposition $p \rightarrow q$ does not depends on the truth value of $q$. The propositions $p \rightarrow q$ and $\neg p$ being $T$ does not concludes that $\neg q$ is also $T$. Hence, it cannot be a deduction rule.


\section*{Answer 5}

\vspace{10px}

\begin{table}[H]
	\centering
	\caption{Answer 5}
	\vspace{5px}
	\begin{tabular}{*6{l}}
		$1.$ & & & $p \rightarrow q$ & \textit{premise} & \\
		
		$2.$ & & & $q \rightarrow r$ & \textit{premise} & \\
		
		$3.$ & & & $r \rightarrow p$ & \textit{premise} & \\
		\cline{2-6}
		
		$4.$ &\multicolumn{1}{|c}{} & & $q$ &\textit{assumption} &\multicolumn{1}{c|}{}\\
		
		$5.$ &\multicolumn{1}{|c}{} & & $r$ &\textit{$\rightarrow$e, 2, 4} &\multicolumn{1}{c|}{}\\
		
		$6.$ &\multicolumn{1}{|c}{} & & $p$ &\textit{$\rightarrow$e, 3, 5} &\multicolumn{1}{c|}{}\\ \cline{2-6}
		
		$7.$ & & & $q \rightarrow p$ & \textit{$\rightarrow$i, 4-6} & \\ \cline{2-6}
		
		$8.$ &\multicolumn{1}{|c}{} & & $p$ &\textit{assumption} &\multicolumn{1}{c|}{}\\
		
		$9.$ &\multicolumn{1}{|c}{} & & $q$ &\textit{$\rightarrow$e, 1, 8} &\multicolumn{1}{c|}{}\\
		
		$10.$ &\multicolumn{1}{|c}{} & & $r$ &\textit{$\rightarrow$e, 2, 9} &\multicolumn{1}{c|}{}\\ \cline{2-6}
		
		$11.$ & & & $p \rightarrow r$ & \textit{$\rightarrow$i, 8-10} & \\
		
		$12.$ & & & $p \leftrightarrow q$ & \textit{$\leftrightarrow$i, 1, 7} & \\
		
		$13.$ & & & $p \leftrightarrow r$ & \textit{$\leftrightarrow$i, 3, 11} & \\
		
		$14.$ & & & $(p \leftrightarrow q) \land (p \leftrightarrow r)$ & \textit{$\land$i, 12, 13} & \\
	
	\end{tabular}
\end{table}

\section*{Answer 6}

\vspace{10px}

\begin{table}[H]
	\centering
	\caption{Answer 6}
	\vspace{5px}
	\begin{tabular}{*6{l}}
		$1.$ & & & $\forall x ( Q(x) \rightarrow R(x))$ & \textit{premise} & \\
		
		$2.$ & & & $\exists x(P(x) \rightarrow Q(x))$ & \textit{premise} & \\
		
		$3.$ & & & $\forall x P(x)$ & \textit{premise} & \\
		\cline{2-6}
		
		$4.$ &\multicolumn{1}{|c}{$c$} & & $P(c) \rightarrow Q(c)$ &\textit{assumption} &\multicolumn{1}{c|}{}\\
		
		$5.$ &\multicolumn{1}{|c}{} & & $P(c)$ &\textit{$\forall$e, 3} &\multicolumn{1}{c|}{}\\
		
		$6.$ &\multicolumn{1}{|c}{} & & $Q(c) \rightarrow R(c)$ &\textit{$\forall$e, 1} &\multicolumn{1}{c|}{}\\
		
		$7.$ &\multicolumn{1}{|c}{} & & $Q(c)$ &\textit{$\rightarrow$e, 4, 5} &\multicolumn{1}{c|}{}\\
		
		$8.$ &\multicolumn{1}{|c}{} & & $R(c)$ &\textit{$\rightarrow$e, 6, 7} &\multicolumn{1}{c|}{}\\
		
		$9.$ &\multicolumn{1}{|c}{} & & $P(c) \land R(c)$ &\textit{$\land$i, 5, 8} &\multicolumn{1}{c|}{}\\
		
		$10.$ &\multicolumn{1}{|c}{} & & $\exists x (P(x) \land R(x)$ &\textit{$\exists$i, 9} &\multicolumn{1}{c|}{}\\ \cline{2-6}
		
		$11.$ & & & $\exists x (P(x) \land R(x))$ & \textit{$\exists$e, 2, 4-10} & \\
		
	\end{tabular}
\end{table}


\end{document}

​
