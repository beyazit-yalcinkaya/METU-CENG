\documentclass[12pt]{article}
\usepackage[utf8]{inputenc}
\usepackage{float}
\usepackage{amsmath}


\usepackage[hmargin=3cm,vmargin=6.0cm]{geometry}
%\topmargin=0cm
\topmargin=-2cm
\addtolength{\textheight}{6.5cm}
\addtolength{\textwidth}{2.0cm}
%\setlength{\leftmargin}{-5cm}
\setlength{\oddsidemargin}{0.0cm}
\setlength{\evensidemargin}{0.0cm}

%misc libraries goes here


\begin{document}

\section*{Student Information } 
%Write your full name and id number between the colon and newline
%Put one empty space character after colon and before newline
Full Name : \ Beyazıt Yalçınkaya \ \\
Id Number : \ 2172138 \ \\

% Write your answers below the section tags
\section*{Answer 1}

\renewcommand{\theenumi}{\alph{enumi}}
\begin{enumerate}
    \item Assume that $\ x \in (A \cap B)$.
        \begin{equation*}
        \begin{split}
        x \in (A \cap B) & \rightarrow x \in A \land x \in B 
        \\
        x \in A \land x \in B & \rightarrow (x \in A \lor x\in \overline{B}) \land (x \in \overline{A} \lor x \in B) 
        \\
        (x \in A \lor x\in \overline{B}) \land (x \in \overline{A} \lor x \in B) & \rightarrow x \in (A \cup \overline{B}) \land x \in (\overline{A} \cup B) 
        \\
        x \in (A \cup \overline{B}) \land x \in (\overline{A} \cup B) & \rightarrow x \in ((A \cup \overline{B}) \cap (\overline{A} \cup B))
        \end{split}
        \end{equation*}
        For an $x$ such that $x \in A \cap B$, we obtained that $x \in ((A \cup \overline{B}) \cap (\overline{A} \cup B))$. Thus, the following has been proven, $A \cap B \subseteq (A \cup \overline{B}) \cap (\overline{A} \cup B)$.
        
    \item Assume that $x \in (\overline{A} \cap \overline{B})$.
        \begin{equation*}
        \begin{split}
        x \in (\overline{A} \cap \overline{B}) & \rightarrow x \in \overline{A} \land x \in \overline{B} 
        \\
        x \in \overline{A} \land x \in \overline{B}  & \rightarrow (x \in \overline{A} \lor x\in B) \land (x \in A \lor x \in \overline{B}) 
        \\
        (x \in \overline{A} \lor x\in B) \land (x \in A \lor x \in \overline{B}) & \rightarrow x \in (\overline{A} \cup B) \land x \in(A \cup \overline{B})
        \\
        x \in (\overline{A} \cup B) \land x \in(A \cup \overline{B}) & \rightarrow  x \in ((A \cup \overline{B}) \cap (\overline{A} \cup B))
        \end{split}
        \end{equation*}
        For an $x$ such that $x  \in \overline{A} \cap \overline{B}$, we obtained that $x \in ((A \cup \overline{B}) \cap (\overline{A} \cup B))$. Thus, the following has been proven, $\overline{A} \cap \overline{B} \subseteq (A \cup \overline{B}) \cap (\overline{A} \cup B)$.
        
 \end{enumerate}



\section*{Answer 2}

In order to prove $f^{-1}((A \cap B) \times C) = f^{-1}(A \times C) \cap f^{-1}(B \times C)$, the following two conditions must be satisfied:
\renewcommand{\theenumi}{\roman{enumi}}
\begin{enumerate}
    \item $f^{-1}((A \cap B) \times C) \subseteq f^{-1}(A \times C) \cap f^{-1}(B \times C)$
    \item $f^{-1}(A \times C) \cap f^{-1}(B \times C) \subseteq f^{-1}((A \cap B) \times C)$
\end{enumerate}
(Note that, for any two set $A$ and $B$, if $A \subseteq B$ and $B \subseteq A$, then we say that $A = B$.)\\
(Also, note that, since $f$ has an inverse $f^{-1}$, $f$ is a one-to-one and an onto function. Thus, any object $x_1$, that we pick from $X$, has a distinct image $f(x_1) = (x_2, x_3)$ from $Y \times Z$ and, similarly, any 2-tuple object $(x_2, x_3)$, that we pick from $Y \times Z$, has a distinct image $f^{-1}(x_2, x_3) = x_1$ from $X$.)

\renewcommand{\theenumi}{\roman{enumi}}
\begin{enumerate}
    \item{$f^{-1}((A \cap B) \times C) \subseteq f^{-1}(A \times C) \cap f^{-1}(B \times C)$ \\ Assume that $f^{-1}(x_1, x_2) \in f^{-1}((A \cap B) \times C)$.
    \begin{equation*}
    \begin{split}
        f^{-1}(x_1, x_2) \in f^{-1}((A \cap B) \times C) \rightarrow & (x_1, x_2) \in ((A \cap B) \times C)\\
        (x_1, x_2) \in ((A \cap B) \times C) \rightarrow & x_1 \in (A \cap B) \land x_2 \in C\\
        x_1 \in (A \cap B) \land x_2 \in C \rightarrow & x_1 \in A \land x_1 \in B \land x_2 \in C\\
        x_1 \in A \land x_1 \in B \land x_2 \in C \rightarrow & (x_1 \in A \land x_2 \in C) \land (x_1 \in B \land x_2 \in C)\\
        (x_1 \in A \land x_2 \in C) \land (x_1 \in B \land x_2 \in C) \rightarrow & (x_1, x_2) \in (A \times C) \land (x_1, x_2) \in (B \times C)\\
        (x_1, x_2) \in (A \times C) \land (x_1, x_2) \in (B \times C) \rightarrow & f^{-1}(x_1, x_2) \in f^{-1}(A \times C) \\ & \qquad \qquad \land f^{-1}(x_1, x_2) \in f^{-1}(B \times C)\\
        f^{-1}(x_1, x_2) \in f^{-1}(A \times C) \land f^{-1}(x_1, x_2) \in f^{-1}(B \times C) \rightarrow & f^{-1}(x_1, x_2) \in f^{-1}(A \times C) \cap f^{-1}(B \times C)\\
    \end{split}
    \end{equation*}}
    For $f^{-1}(x_1, x_2) \in f^{-1}((A \cap B) \times C)$, it has been shown that $f^{-1}(x_1, x_2) \in f^{-1}(A \times C) \cap f^{-1}(B \times C)$. Thus, the following is true, $f^{-1}((A \cap B) \times C) \subseteq f^{-1}(A \times C) \cap f^{-1}(B \times C)$.
    
    \item{$f^{-1}(A \times C) \cap f^{-1}(B \times C) \subseteq f^{-1}((A \cap B) \times C)$ \\ Assume that $f^{-1}(x_1, x_2) \in f^{-1}(A \times C) \cap f^{-1}(B \times C)$.
    \begin{equation*}
    \begin{split}
        f^{-1}(x_1, x_2) \in f^{-1}(A \times C) \cap f^{-1}(B \times C) \rightarrow & f^{-1}(x_1, x_2) \in f^{-1}(A \times C) \\ & \qquad \qquad \land f^{-1}(x_1, x_2) \in f^{-1}(B \times C)\\
        f^{-1}(x_1, x_2) \in f^{-1}(A \times C) \land f^{-1}(x_1, x_2) \in f^{-1}(B \times C) \rightarrow & (x_1, x_2) \in (A \times C) \land (x_1, x_2) \in (B \times C)\\
        (x_1, x_2) \in (A \times C) \land (x_1, x_2) \in (B \times C) \rightarrow & x_1 \in A \land x_2 \in C \land x_1 \in B \land x_2 \in C\\
        x_1 \in A \land x_2 \in C \land x_1 \in B \land x_2 \in C \rightarrow & x_1 \in A \land x_1 \in B \land x_2 \in C\\
        x_1 \in A \land x_1 \in B \land x_2 \in C \rightarrow & x_1 \in (A \cap B) \land x_2 \in C\\
        x_1 \in (A \cap B) \land x \in C \rightarrow & (x_1, x_2) \in ((A \cap B) \times C)\\
        (x_1, x_2) \in ((A \cap B) \times C) \rightarrow & f^{-1}(x_1, x_2) \in f^{-1}((A \cap B) \times C)\\
    \end{split}
    \end{equation*}}
    For $f^{-1}(x_1, x_2) \in f^{-1}(A \times C) \cap f^{-1}(B \times C)$, it has been shown that $f^{-1}(x_1, x_2) \in f^{-1}((A \cap B) \times C)$. Thus, the following is true, $f^{-1}(A \times C) \cap f^{-1}(B \times C) \subseteq f^{-1}((A \cap B) \times C)$.
\end{enumerate}

\noindent
As shown above, both conditions i and ii are true. Thus, the following has been proven:
\begin{equation*}
    f^{-1}((A \cap B) \times C) = f^{-1}(A \times C) \cap f^{-1}(B \times C)    
\end{equation*}





\section*{Answer 3}

\renewcommand{\theenumi}{\alph{enumi}}
\renewcommand{\theenumii}{\roman{enumii}}
\begin{enumerate}
\item
    \begin{enumerate}
        \item Assume that $f(x) = ln(x^2 + 5)$ is a one-to-one function, then by definition the following must be true,\\
        $\forall x \forall y (f(x) = f(y) \rightarrow x = y)$\\
        Let's test it for some $x \in R$ and $y \in R$.\\
        \begin{equation} 
        \label{eq1}
        \begin{split}
            f(x) & = f(y) \\
            ln(x^2 + 5) &= ln(y^2 + 5)\\
            x^2 + 5 &= y^2 + 5\\
            x^2 &= y^2\\
            x = y \ & or \ x = -y
        \end{split}
        \end{equation}
        In Equation (1) put $x = 1$, then both $y = 1$ and $y = -1$ are true, so it is not the case that for all $x$ and $y$ if $f(x) = f(y)$, then $x = y$. It is a contradiction with the assumption. The assumption is discharged; thus $f(x) = ln(x^2 + 5)$ is not a one-to-one function.

        \item For $f(x) = ln(x^2 + 5)$, $f(R) = [1.609, \infty)$, alternatively, $\forall x (x \in R \rightarrow f(x) \in [1.609, \infty))$, note that the minimum value of $x^2$ is $0$, so the minimum value of $ln(x^2 + 5)$ is $ln(5) = 1.609$. Thus, it can be seen that range of $f(x)$ is $[1.609, \infty)$, it is given that co-domain of the function is $R$; thus, the following can be concluded, since $R \neq [1.609, \infty)$, $f(x) = ln(x^2 + 5)$ is not an onto function.
    \end{enumerate}
    As shown above, $f(x) = ln(x^2 + 5)$ is not a one-to-one function and also it is not an onto function.
  
\item
    \begin{enumerate}
        \item Assume that $f(x) = e^{e^{x^7}}$ is a one-to-one function, then by definition the following must be true,\\
        $\forall x \forall y (f(x) = f(y) \rightarrow x = y)$\\
        Let's test it for some $x \in R$ and $y \in R$.\\
        \begin{equation} 
        \label{eq1}
        \begin{split}
            f(x) & = f(y) \\
            e^{e^{x^7}} &= e^{e^{y^7}}\\
            e^{x^7} &= e^{y^7}\\
            x^7 &= y^7\\
            x &= y
        \end{split}
        \end{equation}
        In Equation (2) put $x = c$ for any constant $c \in R$, then only $y = c$ is true, so for all $x$ and $y$ if $f(x) = f(y)$, then $x = y$. The assumption is correct; thus $f(x) = e^{e^{x^7}}$ is a one-to-one function.

        \item For $f(x) = f(x) = e^{e^{x^7}}$, $f(R) = (1, \infty)$, alternatively, $\forall x (x \in R \rightarrow f(x) \in (1, \infty))$, note that, for $x \to -\infty$, $e^{x^7} \to 0$ and for $e^{x^7} \to 0$, $e^{e^{x^7}} \to 1$. Thus, it can be seen that range of $f(x)$ is $(1, \infty)$, it is given that co-domain of the function is $R$; thus, the following can be concluded, since $R \neq (1, \infty)$, $f(x) = ln(x^2 + 5)$ is not an onto function.
    \end{enumerate}
    As shown above, $f(x) = e^{e^{x^7}}$ is a one-to-one function, but it is not an onto function.
    
\end{enumerate}


\section*{Answer 4}

\renewcommand{\theenumi}{\alph{enumi}}
\renewcommand{\theenumii}{\roman{enumii}}
\begin{enumerate}
  \item There are four cases that must be considered.
  
  \begin{enumerate}
    \item A and B are both finite sets.\\
        Assume that $|A| = n$ and $|B| = m$.\\
        $A = \{a_1, a_2, a_3, a_4, ..., a_n\}$ \\
        $B = \{b_1, b_2, b_3, b_4, ..., b_m\}$ \\
        $A \times B = \{(a_1, b_1), (a_1, b_2), ..., (a_1, b_m), (a_2, b_1), (a_2, b_2), ..., (a_2, b_m), ..., (a_n, b_1), (a_n, b_2), ..., (a_n, b_m)\}$ \\
        Since both A and B are finite sets, $A \times B$ is also a finite set with the cardinality $|A \times B| = |A||B| = nm$. Thus, $A \times B$ is a countable set.
    \item A is a countable infinite set and B is a finite set.\\
        Assume that $|B| = m$.\\
        $A = \{a_1, a_2, a_3, a_4, ..., a_m, ...\}$ \\
        $B = \{b_1, b_2, b_3, b_4, ..., b_m\}$ \\
        
        \begin{table}[H]
        \caption{}
        \begin{center}
        \begin{tabular}{ c | c c c c c c c }
                      & $a_1$ & $a_2$ & $a_3$ & $a_4$ & $...$ & $a_m$ & $...$ \\
                \hline
                $b_1$ & $(a_1, b_1)$ & $(a_2, b_1)$ & $(a_3, b_1)$ & $(a_4, b_1)$ & $...$ & $(a_m, b_1)$ & $...$ \\ 
                $b_2$ & $(a_1, b_2)$ & $(a_2, b_2)$ & $(a_3, b_2)$ & $(a_4, b_2)$ & $...$ & $(a_m, b_2)$ & $...$ \\ 
                $b_3$ & $(a_1, b_3)$ & $(a_2, b_3)$ & $(a_3, b_3)$ & $(a_4, b_3)$ & $...$ & $(a_m, b_3)$ & $...$ \\ 
                $b_4$ & $(a_1, b_4)$ & $(a_2, b_4)$ & $(a_3, b_4)$ & $(a_4, b_4)$ & $...$ & $(a_m, b_4)$ & $...$ \\ 
                $...$ & $...$ & $...$ & $...$ & $...$ & $...$ & $...$ & $...$ \\ 
                $b_m$ & $(a_1, b_m)$ & $(a_2, b_m)$ & $(a_3, b_m)$ & $(a_4, b_m)$ & $...$ & $(a_m, b_m)$ & $...$ \\ 
        \end{tabular}
        \end{center}
        \end{table}
        $A \times B = \{ (a_1, b_1), (a_1, b_2), ..., (a_1, b_m), (a_2, b_1), (a_2, b_2), ..., (a_2, b_m), ... \}$\\
        $A \times B$ can be counted column by column in Table 1. Thus, $A \times B$ is a countable set.
        
    \item A is a finite set and B is a countable infinite set.\\
        Same proof as (ii), define A = B and B = A.
    \item A and B are both countable infinite sets.\\
        $A = \{a_1, a_2, a_3, a_4, ...\}$ \\
        $B = \{b_1, b_2, b_3, b_4, ...\}$ \\
        
        \begin{table}[H]
        \caption{}
        \begin{center}
        \begin{tabular}{ c | c c c c c }
                      & $a_1$ & $a_2$ & $a_3$ & $a_4$ & $...$ \\
                \hline
                $b_1$ & $(a_1, b_1)$ & $(a_2, b_1)$ & $(a_3, b_1)$ & $(a_4, b_1)$ & $...$ \\ 
                $b_2$ & $(a_1, b_2)$ & $(a_2, b_2)$ & $(a_3, b_2)$ & $(a_4, b_2)$ & $...$ \\ 
                $b_3$ & $(a_1, b_3)$ & $(a_2, b_3)$ & $(a_3, b_3)$ & $(a_4, b_3)$ & $...$ \\ 
                $b_4$ & $(a_1, b_4)$ & $(a_2, b_4)$ & $(a_3, b_4)$ & $(a_4, b_4)$ & $...$ \\ 
                $...$ & $...$ & $...$ & $...$ & $...$ & $...$ \\
        \end{tabular}
        \end{center}
        \end{table}
        $A \times B = \{ (a_1, b_1), (a_2, b_1), (a_1, b_2), (a_1, b_3), (a_2, b_2), (a_3, b_1), (a_4, b_1), (a_3, b_2), (a_2, b_3), (a_1, b_4), ... \}$\\
        We can enumarate every element in Table 2 by the reverse diagonal of the table, and by doing that we can count $A \times B$. Thus, $A \times B$ is a countable set.
   \end{enumerate}
   
   In conclusion, it has been shown that if $A$ and $B$ are two countable sets, $A \times B$ is also a countable set.
   
   \item Assume that $B$ is a countable set. It is known that any subset of a countable set must be also a countable set (Since any subset $S'$ of a countable set $S$ has at most same number of elements with $S$, so $S$ being countable makes $S'$ a countable set too.), so $A \subseteq B$ is not true for uncountable set $A$, if $B$ is a countable set. It is a contradiction with the assumption. The assumption is discharged; thus, $B$ is an uncountable set.
   
   \item Assume that $A$ is an uncountable set. For countable set $B$, any subset of $B$ is also a countable set (Since any subset $S'$ of a countable set $S$ has at most same number of elements with $S$, so $S$ being countable makes $S'$ a countable set too.), so $A \subseteq B$ is not true for countable set $B$, if $A$ is an uncountable set. It is a contradiction with the assumption. The assumption is discharged; thus, $A$ is a countable set.
   
 \end{enumerate}




\section*{Answer 5}

\renewcommand{\theenumi}{\alph{enumi}}
\begin{enumerate}
    \item By definition $f_1(x)$ is $\mathcal{O}(f_2(x))$ means, for some $c_1$ and $k$, $|f_1(x)| \leq c_1|f_2(x)|$ where $x > k$. Then the following can be done.\\
    For some $c_1$, $k$ and $x > k$,
    \begin{equation}
        \begin{split}
            |f_1(x)| & \leq c_1 |f_2(x)| \\
            ln|f_1(x)| & \leq ln(c_1 |f_2(x)|) \\
            ln|f_1(x)| & \leq ln(c_1) + ln|f_2(x)|\\
            ln|f_1(x)| & \leq c_2 + ln|f_2(x)| \leq c_3ln|f_2(x)| \ \ (for \ some \ c_2 = ln(c_1) \ and \ c_3)\\
            ln|f_1(x)| & \leq c_3ln|f_2(x)|\\
            ln|f_1(x)| & \ is \ \mathcal{O}(ln|f_2(x)|) \\
        \end{split}
    \end{equation}
    As shown above if $f_1(x)$ is $\mathcal{O}(f_2(x))$, then $ln|f_1(x)|$ is $\mathcal{O}(ln|f_2(x)|)$.
    
     \item Let say $f_1(x) = 2x^2 + 10x$ and $f_2(x) = x^2$. Note that $2x^2 + 10x$ and $x^2$ are both increasing functions and $f_1(x)$ is $\mathcal{O}(f_2(x))$ since they satisfy the condition $|f_1(x)| \leq 7|f_2(x)|$ for $x > 2$. Assume that $3^{f_1(x)}$ is $\mathcal{O}(3^{f_2(x)})$, then the following can be done for some $c$, $k$ and $x > k$,
     \begin{equation}
         \begin{split}
            |3^{f_1(x)}| & \leq c |3^{f_2(x)}| \\
            |3^{2x^2 + 10x}| & \leq c |3^{x^2}| \\
            3^{2x^2 + 10x} & \leq c 3^{x^2} \qquad \quad (Since \ both \ are \ positive \ increasing \ functions)\\
            3^{2x^2 + 10x - x^2} & \leq c \\
            3^{x^2 + 10x} & \leq c \\
         \end{split}
     \end{equation}
     Such $c$ cannot be found, so we get a contradiction, assumption is discharged, $3^{f_1(x)}$ is not $\mathcal{O}(3^{f_2(x)})$. Thus, for increasing functions $f_1(x)$ and $f_2(x)$, it has been proven that $3^{f_1(x)}$ is not $\mathcal{O}(3^{f_2(x)})$ with a counter example $f_1(x) = 2x^2 + 10x$ and $f_2(x) = x^2$.
\end{enumerate}



\section*{Answer 6}

\renewcommand{\theenumi}{\alph{enumi}}
\begin{enumerate}
    \item
    \begin{equation}
        \begin{split}
            (3^x - 1) \ mod \ (3^y - 1) & = 3^{x \ mod \ y} - 1\\
            (3^x - 3^{x \ mod \ y}) \ mod \ (3^y - 1) & = 0 \qquad (if \ a \equiv b (mod \ n), \ then \ a - b \equiv 0 (mod \ n)) \\
            (3^y - 1) & | (3^x - 3^{x\ mod \ y})\\
        \end{split}
    \end{equation}
    There are three possibilities for $x$ and $y$ that need to be considered using the information from Equation (5).
    \renewcommand{\theenumii}{\roman{enumii}}
    \begin{enumerate}
        \item if $x = y$, then $x \ mod \ y = 0$.
        \begin{equation}
            \begin{split}
                (3^y - 1) & | (3^x - 3^{x\ mod \ y})\\
                (3^y - 1) & | (3^y - 3^0)\\
                (3^y - 1) & | (3^y - 1) \quad (This \ is \ true \ for \ all \ x \ and \ y)
            \end{split}
        \end{equation}
        \item if $x < y$, then $x \ mod \ y = x$.
        \begin{equation}
            \begin{split}
                (3^y - 1) & | (3^x - 3^{x\ mod \ y})\\
                (3^y - 1) & | (3^x - 3^x)\\
                (3^y - 1) & | 0 \quad (This \ is \ true \ for \ all \ x \ and \ y)
            \end{split}
        \end{equation}
        \item if $x > y$, then $x \ mod \ y = k$ for some positive integer $c$ and $k$ satisfying $x = cy + k$.
        \begin{equation}
            \begin{split}
                (3^y - 1) & | (3^x - 3^{x\ mod \ y})\\
                (3^y - 1) & | (3^{cy + k} - 3^k)\\
                (3^y - 1) & | (3^k(3^{cy} - 1)) \\
                (3^y - 1) & | (3^{cy} - 1) \quad (3^k \ is \ odd \ and \ (3^y - 1) \ is \ even, \ (3^y - 1)|3^k \ is \ impossible) \\
                (3^y - 1) & | (3^{cy} - 1) \quad (This \ is \ true \ for \ all \ x, \ y \ and \ c) \\
            \end{split}
        \end{equation}
    \end{enumerate}
    Thus, it is has been proven that, $(3^x - 1) \mod (3^y - 1) = 3^{x \mod y} - 1$ is true for $x, \ y \in Z^+$.
    
    \item
    \begin{equation}
        \begin{split}
            gcd(123, 277) & = gcd(277, 123)\\
            & = gcd(123, 31)\\
            & = gcd(31, 30)\\
            & = gcd(30, 1)\\
            & = gcd(1, 0)\\
            & = 1
        \end{split}
    \end{equation}
\end{enumerate}

\end{document}

​
