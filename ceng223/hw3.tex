\documentclass[12pt]{article}
\usepackage[utf8]{inputenc}
\usepackage{float}
\usepackage{amsmath}


\usepackage[hmargin=3cm,vmargin=6.0cm]{geometry}
%\topmargin=0cm
\topmargin=-2cm
\addtolength{\textheight}{6.5cm}
\addtolength{\textwidth}{2.0cm}
%\setlength{\leftmargin}{-5cm}
\setlength{\oddsidemargin}{0.0cm}
\setlength{\evensidemargin}{0.0cm}



\begin{document}

\section*{Student Information } 
%Write your full name and id number between the colon and newline
%Put one empty space character after colon and before newline
Full Name : Beyazıt Yalçınkaya \\
Id Number : 2172138 \\

% Write your answers below the section tags
\section*{Answer 1}

    \begin{equation} 
    \label{eq1}
    \begin{split}
        For \ all \ n \geq 1, \Bigg(\sum_{k = 1}^{n} k \Bigg)^2 \geq \sum_{k = 1}^{n} k^2
    \end{split}
    \end{equation}
        
    \subsection*{Basis Step:}
        \qquad    
            For $n = 1$:
        \begin{equation} 
        \label{eq1}
        \begin{split}
            \Bigg(\sum_{k = 1}^{1} k\Bigg)^2 & \geq \sum_{k = 1}^{1} k^2 \\
            (1)^2 & \geq 1^2 \\
            1 & \geq 1
        \end{split}
        \end{equation}
        
        \qquad For $n = 1$, $\Big(\sum_{k = 1}^{n} k\Big)^2 \geq \sum_{k = 1}^{n} k^2$ is true. This completes the basis step.
    
    \subsection*{Inductive Step:}
        \qquad Assume that $\Big(\sum_{k = 1}^{c} k\Big)^2 \geq \sum_{k = 1}^{c} k^2$ is true for an arbitrary fixed integer $c \geq 1$. Under this assumption we need to find if $\Big(\sum_{k = 1}^{c + 1} k\Big)^2 \geq \sum_{k = 1}^{c + 1} k^2$ is true.
        \begin{equation} 
        \label{eq1}
        \begin{split}
            \Bigg(\sum_{k = 1}^{c} k\Bigg)^2 & \geq \sum_{k = 1}^{c} k^2 \qquad (By \ inductive \ hypothesis) \\
            \Bigg(\sum_{k = 1}^{c} k\Bigg)^2 & \geq \sum_{k = 1}^{c} k^2 - 2(c + 1)\Bigg(\sum_{k = 1}^{c} k\Bigg) \quad \Big(Since \ 2(c + 1)\Bigg(\sum_{k = 1}^{c} k\Bigg) \geq 0\Big) \\
            \Bigg(\sum_{k = 1}^{c} k\Bigg)^2 + 2(c + 1)\Bigg(\sum_{k = 1}^{c} k\Bigg) & \geq \sum_{k = 1}^{c} k^2 \\
            \Bigg(\sum_{k = 1}^{c} k\Bigg)^2 + 2(c + 1)\Bigg(\sum_{k = 1}^{c} k\Bigg) + (c + 1)^2 & \geq \sum_{k = 1}^{c} k^2 + (c + 1)^2 \\
            \Bigg(\sum_{k = 1}^{c + 1} k\Bigg)^2 & \geq \sum_{k = 1}^{c + 1} k^2 \\
        \end{split}
        \end{equation}
        
        \qquad By Inductive Hypothesis, $\Big(\sum_{k = 1}^{c + 1} k\Big)^2 \geq \sum_{k = 1}^{c + 1} k^2$ is true. This completes the inductive step.
        
        \qquad Both basis and inductive step have been completed; thus, by mathematical induction, $\Big(\sum_{k = 1}^{n} k\Big)^2 \geq \sum_{k = 1}^{n} k^2$ is true for all $n \geq 1$.
        


\section*{Answer 2}

\subsection*{1.}

\qquad Let's define the set of positive integers up to $41$ as set $S$, then $S = \{1,2,3,...,20,21,22,...,39,40,41\}$. $S$ can be divided into its subsets as following; $\{1, 41\},\ \{2, 40\},\ \{3, 39\},\ ..., \{20, 22\}$. Note that, sum of two elements of each subset is $42$.

\qquad Because players are playing with their best strategies, at one point of the game $21$ will be picked by one of the players, since it does not have a pair such that their sum is $42$.

\qquad Since there are $20$ pairs of the form $\{x_1, x_2\}$ such that $x_1 + x_2 = 42$ and there is number $21$, which is guaranteed to be picked at some point of the game, there will be $22$ picks at most and the $22nd$ pick will be the losing pick, it can proven by the Pigeonhole Principle as following:

\qquad Since $21$ has no pair, we do not consider it for now. Then there will be $21$ picks (excluding number $21$) and there are $20$ subsets to pick from. Then, by Pigeonhole Principle, (pigeons are defined as number of picks and pigeonholes are defined as number of pairs) $\lceil21/20\rceil = 2$, meaning there will be at least one pair whose both elements will be picked.

\qquad Now we include the pick for $21$, so there will be $22$ picks in the game and the $22nd$ pick will be the losing pick. Since Alice started to game, she will always pick at the odd numbered turns and Bob will pick at even numbered turns. Since $22$ is an even number, it will be Bob's turn; thus, Bob will lose the game. Alice will win the game.

\subsection*{2.}

\qquad Since the order does not matter, we need to find different sets with 3 elements and the summation of the elements must be $5$. Since $5$ is not a big number it is trivial to find such sets. The followings are the all of them: $\{5, 0, 0\}, \ \{4, 1, 0\}, \ \{3, 2, 0\}, \ \{3, 1, 1\}, \ \{2, 2, 1\}$. Thus, there are $5$ ways to pick three such integers.

\subsection*{3.}

\qquad In this problem we can think that we have 5 beans and we need to divide them into three, so we can visualize it as following: $* \ * \ * \ * \ * \ | \ |$ (where asterisks are beans and pipes are separators). Thus, we need to place 2 separators. We can find it as following:
        \begin{equation} 
        \label{eq1}
        \begin{split}
            C(7, 2) & = \frac{7!}{5!2!} \\
            & = 21
        \end{split}
        \end{equation}
        
\qquad Note that, we can also directly use the formula for r-combinations where $n$ is defined as $3$ and $r$ is defined as $5$:
        \begin{equation} 
        \label{eq1}
        \begin{split}
            C(n + r - 1, n - 1) & = C(7, 2) \\
            & = \frac{7!}{5!2!} \\
            & = 21
        \end{split}
        \end{equation}
        
Thus, the equation $x_1 + x_2 + x_3 = 5$ has $21$ different solutions.



\section*{Answer 3}

\begin{equation}
\begin{split}
    (1-x^3)^n & = \sum_{k = 0}^{n} a_k x^ k (1-x)^{3n-2k} \\
    \frac{(1-x^3)^n}{(1-x)^n} & = \frac{1}{(1-x)^n}\sum_{k = 0}^{n} a_k x^ k (1-x)^{3n-2k} \\
    \frac{(1-x)^n((1-x)^2+3x)^n}{(1-x)^n} & = \sum_{k = 0}^{n} a_k x^ k \frac{(1-x)^{3n-2k}}{(1-x)^n} \\
    ((1-x)^2+3x)^n & = \sum_{k = 0}^{n} a_k x^ k (1-x)^{2n-2k} \\
    \sum_{k = 0}^n \binom{n}{k} (3x)^k ((1-x)^2)^{n-k} & = \sum_{k = 0}^{n} a_k x^ k ((1-x)^2)^{n-k} \qquad (By \ Binomial \ Theorem) \\
    \sum_{k = 0}^n \binom{n}{k} 3^k x^k ((1-x)^2)^{n-k} & = \sum_{k = 0}^{n} a_k x^ k ((1-x)^2)^{n-k} \\
    a_k & = 3^k \binom{n}{k}\\
    Thus, & \ a_r = 3^r \binom{n}{r}.
\end{split}
\end{equation}


\section*{Answer 4}

We need to find homogeneous solution $a_n^{(h)}$ and particular solution $a_n^{(p)}$, then for the solution $a_n$ we need to find $a_n = a_n^{(h)} + a_n^{(p)}$. Let's first write characteristic equation and find characteristic roots of the equation.
    \begin{equation} 
        \label{eq1}
        \begin{split}
            a_n & = 4a_{n-1} - a_{n-2} - 6a_{n-3} + n - 2 \\
            a_n - 4a_{n-1} + a_{n-2} + 6a_{n-3} & = n - 2 \\
            r^3 - 4r^2 + r + 6 & = 0 \qquad (Characteristic \ Equation) \\
            (r-2)(r-3)(r+1) & = 0
        \end{split}
        \end{equation}
    Thus, characteristic roots are $r = 2$, $r = 3$ and $r = -1$. Then, $a_n^{(h)}$ is of the form $A(2)^n + B(3)^n + C(-1)^n$ and since characteristic roots do not appear as multipliers of $(n-2)$, $a_n^{(p)}$ is of the form $Dn+E$. Now, we can find $A$, $B$, $C$, $D$ and $E$ as following.
    \begin{equation} 
        \label{eq1}
        \begin{split}
            a_n & = 4a_{n-1} - a_{n-2} - 6a_{n-3} + n - 2 \\
            Dn + E & = 4(D(n-1) + E) - (D(n-2)+E) - 6(D(n-3)+E) + n - 2 \\
            n - 2 & = n(4D)+(4E-16D) \\
            D = 0.25 & \ and \ E = 0.50\\
            a_n & = a_n^{(h)} + a_n^{(p)} \\
            a_n & = A(2)^n + B(3)^n + C(-1)^n + (0.25)n + 0.50 \\
            a_0 & = A + B + C + 0.50 = 3.5 \\
            a_1 & = 2A + 3B - C + 0.25 + 0.50 = 4.75 \\
            a_2 & = 4A + 9B + C + 2(0.25) + 0.50 = 13 \\
            A & = 1.67, \ B = 0.50 \ and \ C = 0.83 \\
            a_n & = (1.67)(2)^n + (0.50)(3)^n + (0.83)(-1)^n + (0.25)n + 0.50 \\
        \end{split}
        \end{equation}
    Hence, the solution of the given recurrence relation is
    \begin{equation*}
        a_n = (1.67)(2)^n + (0.50)(3)^n + (0.83)(-1)^n + (0.25)n + 0.50
    \end{equation*}




\end{document}

​
