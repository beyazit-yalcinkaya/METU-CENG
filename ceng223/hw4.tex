\documentclass[12pt]{article}
\usepackage[utf8]{inputenc}
\usepackage{float}
\usepackage{amsmath}
\usepackage{tikz} % for Hasse diagram
\usepackage[hmargin=3cm,vmargin=6.0cm]{geometry}
%\topmargin=0cm
\topmargin=-2cm
\addtolength{\textheight}{6.5cm}
\addtolength{\textwidth}{2.0cm}
%\setlength{\leftmargin}{-5cm}
\setlength{\oddsidemargin}{0.0cm}
\setlength{\evensidemargin}{0.0cm}

\begin{document}
	
\section*{Student Information } 
%Write your full name and id number between the colon and newline
%Put one empty space character after colon and before newline
Full Name : Beyazıt Yalçınkaya \\
Id Number : 2172138 \\

% Write your answers below the section tags
\section*{Answer 1}

Let $a_n$ be the number of ternary strings that contain $3$ consecutive $0s$, $1s$ or $2s$. The following cases must be considered in order to find the recurrence relation for $a_n$.
\begin{itemize}
    \item $. \ . \ . \ . \ . \ . \ 0 \ 0 \ 0$ \\
        Other than last three $0s$ there are $n-3$ numbers in the string and they can be in any order since with the last three $0s$ the property of $a_n$ has already been satisfied. The numbers before last three $0s$ can be in $3^{n - 3}$ different form, so there are $3^{n - 3}$ different ternary strings. Notice that, instead of $0$, $1$ or $2$ can also be picked in order to form such string, so in overall, there are $3$ cases to consider (ending with three $0s$, ending with three $1s$, and ending with three $2s$). Thus, there are $3 \times 3^{n - 3}$ different ternary strings in the given form.
    \item $. \ . \ . \ . \ . \ . \ . \ 0 \ 0$ \\
        The last two terms of the string do not contain three consecutive $0s$, $1s$ or $2s$ if the term before them is not $0$. Thus, in order this string to satisfy $a_n$, the substring that contains the terms before the last two must contain three consecutive $0s$, $1s$ or $2s$. The number of different such strings with $n - 2$ terms is $a_{n - 2}$. Note that, the substring with $n - 2$ terms should not end with $0$ (because if it ended with $0$, the previous case would be counted again). Among the $a_{n - 2}$ different such strings $\frac{2}{3} \times a_{n - 2}$ of them do not end with $0$. Thus, there are $\frac{2}{3} \times a_{n - 2}$ different strings that satisfies $a_n$ and is in the given form. Note that, like in the previous case, instead of $0$, $1$ or $2$ can also be picked in order to form such string, so there are again $3$ cases to consider and each of them forms $\frac{2}{3} \times a_{n - 2}$ different strings. Thus, there are $3 \times \frac{2}{3} \times a_{n - 2}$ different ternary strings in the given form.
    \item $. \ . \ . \ . \ . \ . \ . \ . \ 0$ \\
        The last term of the string does not contain three consecutive $0s$, $1s$ or $2s$ if the term before that is not $0$. Thus, in order this string to satisfy $a_n$, the substring that contains the terms before the last one must contain three consecutive $0s$, $1s$ or $2s$. The number of different such strings with $n - 1$ terms is $a_{n - 1}$. Note that, the substring with $n - 1$ terms should not end with $0$ (because if it ended with $0$, the previous case would be counted again). Among the $a_{n - 1}$ different such strings $\frac{2}{3} \times a_{n - 1}$ of them do not end with $0$. Thus, there are $\frac{2}{3} \times a_{n - 1}$ different strings that satisfies $a_n$ and is in the given form. Note that, like in the previous cases, instead of $0$, $1$ or $2$ can also be picked in order to form such string, so there are again $3$ cases to consider and each of them forms $\frac{2}{3} \times a_{n - 1}$ different strings. Thus, there are $3 \times \frac{2}{3} \times a_{n - 1}$ different ternary strings in the given form.
\end{itemize}
There is no other case to consider. The final step is adding these different cases up in order to form $a_n$. Thus, the recurrence relation for the number of ternary strings that contain 3 consecutive $0s$, $1s$ or $2s$ is the following.
\begin{equation*}
    \begin{split}
        a_n & = 3 \times \frac{2}{3} \times a_{n - 1} + 3 \times \frac{2}{3} \times a_{n - 2} + 3 \times 3^{n - 3} \\
        a_n & = 2a_{n - 1} + 2a_{n - 2} + 3^{n - 2} \quad (n \geq 3, \ a_0 = 0, \ a_1 = 0, \ a_2 = 0)
    \end{split}
\end{equation*}


\section*{Answer 2}

\renewcommand{\theenumi}{\alph{enumi}}
\renewcommand{\theenumii}{\roman{enumii}}
\begin{enumerate}
    \item Followings are the different ways of covering a $3 \times 2$ board using $2 \times 1$ tiles. \\
    \\
        \begin{tikzpicture}
            \draw (0,0) -- (1,0) -- (1,2) -- (0,2) -- (0,0);
            \draw (1,0) -- (2,0) -- (2,2) -- (1,2) -- (1,0);
            \draw (0,-1) -- (2,-1) -- (2,0) -- (0,0) -- (0,-1);
            
            \draw (5,-1) -- (6,-1) -- (6,1) -- (5,1) -- (5,-1);
            \draw (6,-1) -- (7,-1) -- (7,1) -- (6,1) -- (6,-1);
            \draw (5,1) -- (7,1) -- (7,2) -- (5,2) -- (5,1);
            
            \draw (10,-1) -- (12,-1) -- (12,0) -- (10,0) -- (10,-1);
            \draw (10,0) -- (12,0) -- (12,1) -- (10,1) -- (10,0);
            \draw (10,1) -- (12,1) -- (12,2) -- (10,2) -- (10,1);
        \end{tikzpicture}


Thus, as shown above, there are $3$ ways to cover a $3 \times 2$ board using $2 \times 1$ tiles.

    \item Let's first name the tiles found in part a. for the sake of simplicity. The first tile is named as $b$, the second tile is named as $o$, the third tile is named as $p$. Now $a_n$ can be found as follows.
    
    \begin{enumerate}
        \item Assume that $n$ is an odd number. Then since $3$ is also an odd number, multiplication of these two will be also an odd number, so in the board there will be odd numbered squares. The tiles that are used to cover this board has $2 \times 3 = 6$ squares which is an even number. There is no way to divide an odd number by an even number without any remainder, so no matter how the board is tiled there will be uncover squares, this contradicts with the definition of covering a board. Hence, if n is an odd number there will be zero ways to cover the board, meaning $a_n = 0$.
        
        \item Assume that $n$ is an even number. Before, starting note that, since we want to cover the board, each tile's side with length $3$ must cover the side of the board with length $3$, so that there will not be any uncover row. $a_n$ can be found as follows.
        
        Note that, in the equations below $\frac{n}{2}$ is used, since each placement of a tile means reducing $n$ by $2$ and also there are $\frac{n}{2}$ places to put the tiles.

\begin{equation*}
  \left.\begin{aligned}
  &b \ . \ . \ . \ . \ . \ . \ . \ . \ . \ . \ . \ . \\
  &o \ . \ . \ . \ . \ . \ . \ . \ . \ . \ . \ . \ .\\
  &p \ . \ . \ . \ . \ . \ . \ . \ . \ . \ . \ . \ .
\end{aligned}\right\} Counting \ all \ possibilities, \ 3^{\frac{n}{2}} 
\end{equation*}

\begin{equation*}
  \left.\begin{aligned}
  &b \ . \ . \ . \ . \ . \ . \ . \ . \ . \ . \ . \ . \\
  &p \ . \ . \ . \ . \ . \ . \ . \ . \ . \ . \ . \ .
\end{aligned}\right\} Same \ thing \ one \ of\ them \ must \ be \ subtracted, \ 3^{\frac{n}{2}-1}
\end{equation*}

\begin{equation*}
  \left.\begin{aligned}
  &o \ b \ . \ . \ . \ . \ . \ . \ . \ . \ . \ . \ . \\
  &o \ p \ . \ . \ . \ . \ . \ . \ . \ . \ . \ . \ .
\end{aligned}\right\} Same \ thing \ one \ of\ them \ must \ be \ subtracted, \ 3^{\frac{n}{2}-2}
\end{equation*}

Keep going in that way...


\begin{equation*}
  \left.\begin{aligned}
  &o \ o \ o \ . \ . \ . \ . \ . \ . \ . \ . \ o \ b \\
  &o \ o \ o \ . \ . \ . \ . \ . \ . \ . \ . \ o \ p
\end{aligned}\right\} Same \ thing \ one \ of\ them \ must \ be \ subtracted, \ 3^{0}
\end{equation*}

We need to subtract each repeating term from all possibilities.

\begin{equation*}
    \begin{split}
        a_n = & 3^{\frac{n}{2}} - 3^{\frac{n}{2}-1} - 3^{\frac{n}{2}-2}-. \ . \ .-1
    \end{split}
\end{equation*}

Equation obtained above can be manipulated as follows.

\begin{equation*}
    \begin{split}
        a_n = & 3^{\frac{n}{2}} - 3^{\frac{n}{2}-1} - 3^{\frac{n}{2}-2}-. \ . \ .-1\\
        a_{n-2} = & 3^{\frac{n}{2}-1} - 3^{\frac{n}{2}-2} - 3^{\frac{n}{2}-3}-. \ . \ .-1\\
        3a_{n-2} = & 3^{\frac{n}{2}} - 3^{\frac{n}{2}-1} - 3^{\frac{n}{2}-2}-. \ . \ .-3\\
        a_n = & 3a_{n-2} - 1
    \end{split}
\end{equation*}

Thus, the recurrence relation is $a_n = 3a_{n-2} - 1$ for even $n$'s.
    \end{enumerate}
    
$$
a_n = \left\{
        \begin{array}{ll}
            3a_{n-2} - 1 & \quad for \ n = 2,4,6,8,... \ and \ a_0 = 1 \\
            0 & \quad for \ n = 1,3,5,7,...
        \end{array}
    \right.
$$
    
    \item If $n$ is an odd number the solution is trivial, it is zero. We need to focus on $n = 2,4,6,8,...$. Solution is as follows.
    
\begin{equation*}
    \begin{split}
        a_n & = 3a_{n-2}-2 \quad (n = 2,4,6,8,..., \ a_0 = 1) \\
        G(x) & = \sum_{k = 0}^{\infty}{a_{2k}x^k}\\
        xG(x) & = x\sum_{k = 0}^{\infty}{a_{2k}x^k}\\
        xG(x) & = \sum_{k = 0}^{\infty}{a_{2k}x^{k+1}}\\
        xG(x) & = \sum_{k = 1}^{\infty}{a_{2k-2}x^k}\\
        G(x) - 3xG(x) & = \sum_{k = 0}^{\infty}{a_{2k}x^k} - 3 \sum_{k = 1}^{\infty}{a_{2k-2}x^k}\\
        (1-3x)G(x) & = a_0x^0 + \sum_{k = 1}^{\infty}{x^k(a_{2k}-a_{2k-2})}\\
        (1-3x)G(x) & = 1 - \sum_{k = 1}^{\infty}{x^k} \qquad \ (Since \ a_n - a_{n-2} = -1)\\
        (1-3x)G(x) & = 2 - \frac{1}{1-x}\\
        (1-3x)G(x) & = \frac{1-2x}{1-x}\\
        G(x) & = \frac{1-2x}{(1-x)(1-3x)}\\
        G(x) & = \frac{1}{2}\frac{1}{(1-x)}+\frac{1}{2}\frac{1}{(1-3x)}\\
        G(x) & = \frac{1}{2}\sum_{k=0}^{\infty}{x^k}+\frac{1}{2}\sum_{k=0}^{\infty}{3^kx^k}\\
        G(x) & = \sum_{k=0}^{\infty}{\frac{1}{2}(1+3^k)x^k}\\
        \sum_{k = 0}^{\infty}{a_{2k}x^k} & = \sum_{k=0}^{\infty}{\frac{1}{2}(1+3^k)x^k}\\
        a_{2k} & = \frac{1}{2}(1+3^k)\\
        a_{n} & = \frac{1}{2}(1+3^{\frac{n}{2}}) \qquad for \ n = 0,2,4,6,8,...\\
    \end{split}
\end{equation*}

Thus, $a_n$ is the following.

$$
a_n = \left\{
        \begin{array}{ll}
            \frac{1}{2}(1+3^{\frac{n}{2}}) & \quad for \ n = 0,2,4,6,8,...\\
            0 & \quad for \ n = 1,3,5,7,...
        \end{array}
    \right.
$$


\end{enumerate}


\section*{Answer 3}

\renewcommand{\theenumi}{\alph{enumi}}
\renewcommand{\theenumii}{\roman{enumii}}
\begin{enumerate}
    \item Let $S$ denote any set of sets.
    \begin{enumerate}
        \item $A \subseteq A$ for all $A \in S$. Hence, set inclusion is reflexive on $S$.
        \item $A \subseteq B \land B \subseteq A \rightarrow A = B$ for all $A, B \in S$. Hence, set inclusion is antisymmetric on $S$.
        \item $A \subseteq B \land B \subseteq C \rightarrow A \subseteq C$ for all $A, B, C \in S$. Hence, set inclusion is transitive on $S$.
    \end{enumerate}
    Since, set inclusion is reflexive, antisymmetric, and transitive on any set of sets $S$, it is a partial ordering on $S$, and $(S, \subseteq)$ is a poset.
    
    \item
    \begin{enumerate}
        \item $a | a$ for all $a \in Z$. Hence, $|$ is reflexive on $Z$.
        \item $a | b \land b | a \rightarrow a = \pm b$ for all $a, b \in Z$. Hence, $|$ is not antisymmetric on $Z$.
        \item $a | b \land b | c \rightarrow a | c$ for all $a, b, c \in Z$. Hence, $|$ is transitive on $Z$.
    \end{enumerate}
    Since, $|$ is not antisymmetric on set of integers $Z$, it is not a partial ordering on $Z$.
    
    \item
    \begin{enumerate}
        \item $a \ R \ a$ for all $a \in Z$, since $a = a^1$ and $1 \in Z^+$. Hence, $R$ is reflexive on $Z$.
        \item Say $a \ R \ b$ and $b \ R \ a$ for some $a, b \in Z - \{-1, 0, 1\}$. Then, $a = b^{r_1}$ and $b = a^{r_2}$ for some $r_1, r_2 \in Z^+$. $b$ can be replaced with $a^{r_2}$ in $a = b^{r_1}$ which becomes $a = a^{r_1r_2}$ after the replacement. This implies $r_1r_2 = 1$ and since both $r_1$ and $r_2$ are positive integers, $r_1 = 1$ and $r_2 = 1$. Placing the obtained values of $r_1$ and $r_2$, $a = b^1$ and $b = a^1$ which implies $a = b$. Now, assume that $a \ R \ b$ and $b \ R \ a$ for some $a, b \in \{-1, 0, 1\}$. Then, $a = b^{r_1}$ and $b = a^{r_2}$ for some $r_1, r_2 \in Z^+$. $b$ can be replaced with $a^{r_2}$ in $a = b^{r_1}$ which becomes $a = a^{r_1r_2}$ after the replacement. If $a = -1$, then $r_1r_2$ must be odd, in order $r_1r_2$ to be odd both $r_1$ and $r_2$ must be odd. Since both $r_1$ and $r_2$ are odd, $b = -1$, which means $a = b$. If $a = 0$, then the relation hold only for $b = 0$, regardless the value of $r_1r_2$. Then, $a = b$. If $a = 1$, then since both $r_1$ and $r_2$ are positive integers, $b = 1$. Then, $a = b$. From this it can be concluded that $(a \ R \ b) \land (b \ R \ a) \rightarrow (a = b)$ for all $a, b \in Z$. Hence, $R$ is antisymmetric on $Z$.
        \item Say $a \ R \ b$ and $b \ R \ c$ for some $a, b, c \in Z$. Then, $a = b^{r_1}$ and $b = c^{r_2}$ for some $r_1, r_2 \in Z^+$. $b$ can be replaced with $c^{r_2}$ in $a = b^{r_1}$ which becomes $a = c^{r_1r_2}$ after the replacement. Since both $r_1$ and $r_2$ are positive integers their product will be also a positive integer, this implies $r_1r_2 = r_3$ for some $r_3 \in Z^+$. Then, since $a = c^{r_3}$, $a \ R \ c$. From this it can be concluded that $(a \ R \ b) \land (b \ R \ c) \rightarrow (a \ R \ c)$ for all $a, b, c \in Z$. Hence, $R$ is transitive on $Z$.
    \end{enumerate}
    Since, $R$ is reflexive, antisymmetric, and transitive on set of integers $Z$, it is a partial ordering on $Z$, and $(Z, R)$ is a poset.
    
\end{enumerate}


\section*{Answer 4}

\renewcommand{\theenumi}{\alph{enumi}}
\begin{enumerate}
    \item The partitions of $5$ are as follows.
    \begin{equation*}
        \begin{split}
            & 5 \\
            & 4 + 1 \\
            & 3 + 2 \\
            & 3 + 1 + 1 \\
            & 2 + 2 + 1 \\
            & 2 + 1 + 1 + 1 \\
            & 1 + 1 + 1 + 1 + 1 \\
        \end{split}
    \end{equation*}
    \item The figure below is the Hasse Diagram of the partitions of 5 ordered by precedence.
\end{enumerate}

%Hasse diagram example
\begin{figure}[H]
\centering
\begin{tikzpicture}
%%   kw   (name)   (x, y)   {text}
	\node (n7) at (2, 6)     {5};
	\node (n6) at (0, 5)     {4 + 1};
	\node (n5) at (4, 5)     {3 + 2};
	\node (n4) at (0, 3)     {3 + 1 + 1};
	\node (n3) at (4, 3)     {2 + 2 + 1};
	\node (n2) at (2, 2)     {2 + 1 + 1 + 1};
	\node (n1) at (2, 0)     {1 + 1 + 1 + 1 + 1};

    \draw (n2) -- (n1);
    \draw (n3) -- (n2);
    \draw (n4) -- (n2);
    \draw (n5) -- (n3);
    \draw (n6) -- (n4);
    \draw (n6) -- (n3);
    \draw (n5) -- (n4);
    \draw (n6) -- (n7);
    \draw (n5) -- (n7);
\end{tikzpicture} \caption{Diagram of $\{1 + 1 + 1 + 1 + 1, 2 + 1 + 1 + 1, 2 + 2 + 1, 3 + 1 + 1, 3 + 2, 4 + 1, 5\}$ ordered by precedence.}
\end{figure}

\end{document}