\documentclass[10pt,a4paper, margin=1in]{article}
\usepackage{fullpage}
\usepackage{amsfonts, amsmath, pifont}
\usepackage{amsthm}
\usepackage{graphicx}
\usepackage{fullpage}
\usepackage{amsfonts, amsmath, pifont}
\usepackage{amsthm}
\usepackage{graphicx}
\usepackage{float}

\usepackage{tkz-euclide}
\usepackage{tikz}
\usepackage{pgfplots}
\pgfplotsset{compat=1.13}
\usepackage[utf8]{inputenc}

\usepackage{geometry}
 \geometry{
 a4paper,
 total={210mm,297mm},
 left=10mm,
 right=10mm,
 top=10mm,
 bottom=16mm,
 }
 
 \begin{filecontents}{q2_a.dat}
 n   xn 
 -2   0
 -1   0
 0   1
 1   -1
 2   -8
 3   11
 4   -3
 5   0
 6   0
\end{filecontents}
 
 % Write both of your names here. Fill exxxxxxx with your ceng mail address.
\author{
  Düzel, Uğur\\
  \texttt{e2171569@ceng.metu.edu.tr}
  \and
  Yalçınkaya, Beyazıt\\
  \texttt{e2172138@ceng.metu.edu.tr}
}
\title{CENG 384 - Signals and Systems for Computer Engineers \\
Spring 2018-2019 \\
Written Assignment 2}
\begin{document}
\maketitle



\noindent\rule{19cm}{1.2pt}

\begin{enumerate}

\item 
    \begin{enumerate}
    % Write your solutions in the following items.
    \item The differential equation represented by the given system is presented below. In the second step, we further simplify the expression for the sake of presentation.
    	\begin{equation}
	\begin{split}
		\int_{-\infty}^{t} x(\tau) - 4y(\tau) d \tau & = y(t)\\
		\frac{dy(t)}{dt} + 4y(t) & = x(t)
	\end{split}
	\end{equation}
    \item %write the solution of q1b
	For the input $x(t)=(e^{-t}+e^{-2t})u(t)$ we have the following differential equation 
	\begin{equation}
	y'(t) + 4y(t) = e^{-t}+e^{-2t} \text{\quad for } t > 0 
	\end{equation}
	Since it is not a homogeneous differential equation the solution should be in the from of $y(t)=y_H(t)+y_P(t)$. \\
	To obtain $y_H(t)$ part of the solution we need to solve the following equation
	\begin{equation}
	 y'(t) + 4y(t) =0
	\end{equation}
	We hypothesize the solution as $y_H(t)=Ae^{st}$ and plug this into the differential equation. \\
    	\begin{equation}
	\begin{split}
		Ase^{st} + 4Ase^{st} = 0 & \\
		Ae^{st} (s + 4) = 0 & \\
		s = -4 & \\
		y_H(t) = Ae^{-4t} &\text{\quad for } t>0 
	\end{split}
	\end{equation}	
	To obtain $y_P(t )$ we first should divide it into two for simplicity as $y_P(t)=y_{P_1}(t)+y_{P_2}(t)$. \\ \\
	$y_{P_1}(t)$ is a particular solution for 
	\begin{equation}
	y'(t) + 4y(t) = e^{-t} \text{\quad for } t>0 
	\end{equation}
	$y_{P_2}(t)$ is a particular solution for 	
	\begin{equation}
	y'(t) + 4y(t) = e^{-2t} \text{\quad for } t>0 
	\end{equation}
    	$y_P(t)=y_{P_1}(t)+y_{P_2}(t)$ is a particular solution for 
	\begin{equation}
	y'(t) + 4y(t) = e^{-t}+e^{-2t} \text{\quad for } t>0 
	\end{equation}
	We hypothesize $y_{P_1}(t)=Be^{-t}$ and plug it into 
	\begin{equation}
	\begin{split}
	y'(t) + 4y(t) & = e^{-t} \text{\quad for } t>0 \\
	-Be^{-t}+4Be^{-t} & = e^{-t} \\
	B &= \frac{1}{3}
	\end{split}
	\end{equation}
	We hypothesize $y_{P_2}(t)=Ce^{-2t}$ and plug it into 
	\begin{equation}
	\begin{split}
	y'(t) + 4y(t) & = e^{-t2} \text{\quad for } t>0 \\
	-2Ce^{-2t}+4Ce^{-2t} & = e^{-2t} \\
	C &= \frac{1}{2}
	\end{split}
	\end{equation}
	We found $y_P(t)=y_{P_1}(t)+y_{P_2}(t)=\frac{e^{-t}}{3}+\frac{e^{-2t}}{2}$ for $t>0$ therefore,
	\begin{equation}
	y(t) = Ae^{-4t} + \frac{e^{-t}}{3}+\frac{e^{-2t}}{2}
	\end{equation}
	We assume that the system is initially at rest therefore we say that $y(0)=0$ which yields to
	\begin{equation}
	\begin{split}
	y(0) & = A + \frac{1}{3} + \frac{1}{2} = 0 \\
	A & = - \frac{5}{6} \\
	y(t) & = - \frac{5e^{-4t}}{6}  + \frac{e^{-t}}{3}+\frac{e^{-2t}}{2} \text{\quad for } t>0\\
	y(t) & = \left( - \frac{5e^{-4t}}{6}  + \frac{e^{-t}}{3}+\frac{e^{-2t}}{2} \right) u(t)
	\end{split}
	\end{equation}
    \end{enumerate}


\item %write the solution of q2
    \begin{enumerate}
    \item Below, we give the solution and the graph for $y[n]$.
    	\begin{equation}
	\begin{split}
		x[n] & = \delta[n - 1] - 3 \delta[n - 2] + \delta[n - 3]\\
		h[n] & = \delta[n + 1] + 2 \delta[n] - 3 \delta[n - 1]\\
		y[n] & = x[n] \ast h[n] = \sum_{-\infty}^{+\infty} x[k] h[n - k]\\
		y[0] & = \ldots + x[1]h[0-1] + x[2]h[0-2] + x[3]h[0-3] + \ldots = \ldots + (1)(1) + (-3)(0) + (1)(0) = 1 \\
		y[1] & = \ldots + x[1]h[1-1] + x[2]h[1-2] + x[3]h[1-3] + \ldots = \ldots + (1)(2) + (-3)(1) + (1)(0) = -1 \\
		y[2] & = \ldots + x[1]h[2-1] + x[2]h[2-2] + x[3]h[2-3] + \ldots = \ldots + (1)(-3) + (-3)(2) + (1)(1) = -8 \\
		y[3] & = \ldots + x[1]h[3-1] + x[2]h[3-2] + x[3]h[3-3] + \ldots = \ldots + (1)(0) + (-3)(-3) + (1)(2) = 11 \\
		y[4] & = \ldots + x[1]h[4-1] + x[2]h[4-2] + x[3]h[4-3] + \ldots = \ldots + (1)(0) + (-3)(0) + (1)(-3) = -3
	\end{split}
	\end{equation}
    
    \begin{figure} [H]
    \centering
    \begin{tikzpicture}[scale=1.2] 
      \begin{axis}[
          axis lines=middle,
          xlabel={$\boldsymbol{n}$},
          ylabel={$\boldsymbol{y[n]}$},
          xtick={ -2, -1, 0, ..., 6},
          ytick={-8, -7, -6, -5, ..., 11},
          ymin=-8, ymax=11,
          xmin=-2, xmax=6,
          every axis x label/.style={at={(ticklabel* cs:1.05)}, anchor=west,},
          every axis y label/.style={at={(ticklabel* cs:1.05)}, anchor=south,},
          grid,
        ]
        \addplot [ycomb, black, thick, mark=*] table [x={n}, y={xn}] {q2_a.dat};
      \end{axis}
    \end{tikzpicture}
    \caption{$n$ vs $y[n]$.}
    \label{fig:q3}
\end{figure}
    
    \item Below, we give the solution for $y(t)$.
    	\begin{equation}
	\begin{split}
		x(t) & = u(t) + u(t - 1)\\
		h(t) & = e^{-2t}\cos(t)u(t)\\
		y(t) & = \frac{dx(t)}{dt} \ast h(t)\\
		& = \frac{d(u(t) + u(t - 1))}{dt} \ast h(t)\\
		& = \left( \frac{du(t)}{dt} + \frac{du(t - 1)}{dt} \right) \ast h(t)\\
		& = \left( \delta(t) + \delta(t - 1) \right) \ast h(t)\\
		& = \left( \delta(t) \ast h(t) \right) + \left( \delta(t - 1) \ast h(t) \right)\\
		& = h(t) + h(t - 1)\\
		& = e^{-2t}\cos(t)u(t) + e^{-2(t - 1)}\cos(t - 1)u(t - 1)
	\end{split}
	\end{equation}
    
    
    \end{enumerate}

\item      
    \begin{enumerate}
    \item Below, we give the solution for $y(t)$.
    	\begin{equation}
	\begin{split}
		x(t) & = e^{-t}u(t)\\
		h(t) & = e^{-3t}u(t)\\
		y(t) & = x(t) \ast h(t)\\
		& = \int_{-\infty}^{+\infty} e^{-\tau}u(\tau)e^{-3(t - \tau)}u(t - \tau)d\tau\\
		& = \int_{0}^{+\infty} e^{-\tau}e^{-3(t - \tau)}u(t - \tau)d\tau\\
		& = \int_{0}^{t} e^{-\tau}e^{-3(t - \tau)}d\tau\\
		& = \int_{0}^{t} e^{-\tau}e^{-3t + 3\tau}d\tau\\
		& = \int_{0}^{t} e^{-3t}e^{2\tau}d\tau\\
		& = e^{-3t} \int_{0}^{t} e^{2\tau}d\tau\\
		& = e^{-3t} \left( \frac{e^{2t}}{2} - \frac{1}{2} \right)\\
		& = \frac{e^{-t}}{2} - \frac{e^{-3t}}{2} \\ 
	\end{split}
	\end{equation}
    \item In this question, we need to consider three cases for the values of $t$ while computing the convolution, i.e., $t < 1$, $1 \leq t \leq 2$, and $2 < t$.
    	\begin{equation}
		y(t) = \begin{cases}  0 & \mbox{if } 1 < t \\ 
						\int_{1}^{t} e^{t - \tau} d\tau & \mbox{if } 1 \leq t \leq 2 \\
						 \int_{1}^{2} e^{t - \tau} d\tau & \mbox{if } 2 < t \end{cases}
	\end{equation}
	When we evaluate the integrals, we get the following result.
    	\begin{equation}
		y(t) = \begin{cases}  0 &\mbox{if } 1 < t \\ 
						 e^{t - 1} - 1 & \mbox{if } 1 \leq t \leq 2 \\
					 	 e^{t - 1} - e^{t - 2}& \mbox{if } 2 < t \end{cases}
	\end{equation}
    \end{enumerate}

\item 
    \begin{enumerate}
    \item %write the solution of q4a
    The characteristic equation of this equation is the following
    \begin{equation}
	\begin{split}
		r^2-15r+26 &= 0 \\
		(r - 2) (r - 13) &= 0 \\
		r_1 = 2, \ 	r_2 & = 13
	\end{split}	
    \end{equation}
    Therefore,
    \begin{equation}
	\begin{split}
		y[n] & = A(2^n) + B(13^n) \\
		y[0] & = A + B = 10 \\
		y[1] & = 2A + 13B = 42 \\
		A & = 8 \\
		B & = 2 
	\end{split}	
    \end{equation}
    So the solution is
    \begin{equation}
	y[n]  = 2^{n+3} + 2(13^n) 
    \end{equation}
    \item %write the solution of q4b
    The characteristic equation of this equation is the following
    \begin{equation}
	\begin{split}
		r^2-3r+1 &= 0 \\
		b^2 - 4ac & = 9-4 = 5 > 0 \\ 
		r_1 = \frac{-b+\sqrt{b^2-4ac}}{2a} & = \frac{3+\sqrt{5}}{2} \\
		r_2 = \frac{-b-\sqrt{b^2-4ac}}{2a} & = \frac{3-\sqrt{5}}{2} 
	\end{split}	
    \end{equation}
    Therefore,
    \begin{equation}
	\begin{split}
		y[n] & = A\left(\frac{3+\sqrt{5}}{2}\right)^n + B\left(\frac{3-\sqrt{5}}{2}\right)^n \\
		y[0] & = A + B = 1 \\
		y[1] & = A\left(\frac{3+\sqrt{5}}{2}\right) + B\left(\frac{3-\sqrt{5}}{2}\right) = 2 \\ 
		\end{split}	
    \end{equation}
	When we solve these equations,
    \begin{equation}
	\begin{split}		
		& \frac{3}{2}(A+B)+\frac{\sqrt{5}}{2}(A-B)  =2 \\
		& \frac{\sqrt{5}}{2}(A-B) = \frac{1}{2} \\
		 A - B  & = \frac{\sqrt{5}}{5} \\
		A  & = \frac{5+\sqrt{5}}{10} \\
		B &  = \frac{5-\sqrt{5}}{10}  
	\end{split}	
    \end{equation}
    So the solution is
    \begin{equation}
	y[n]  = \left(\frac{5+\sqrt{5}}{10}\right)\left(\frac{3+\sqrt{5}}{2}\right)^n + \left(\frac{5-\sqrt{5}}{10}\right)\left(\frac{3-\sqrt{5}}{2}\right)^n
    \end{equation}

    \end{enumerate}

\item 
    \begin{enumerate}
    \item Below, we find the impulse response $h(t)$. First, we find the unit step response $s(t)$, then differentiate it to find $h(t)$ since $\frac{ds(t)}{dt} = h(t)$.
    	\begin{equation}
	\begin{split}
		\frac{d^2y(t)}{dt^2} + 	6\frac{dy(t)}{dt} + 8y(t) = 2x(t)
	\end{split}
	\end{equation}
	Assume the homogenous solution is of the form $y_h(t) = Ke^{\alpha t}$. From this, we find the homogenous solution $y_h(t)$.
    	\begin{equation}
	\begin{split}
		\frac{d^2y(t)}{dt^2} + 	6\frac{dy(t)}{dt} + 8y(t) & = 0\\
		\alpha^2Ke^{\alpha t} + 6\alpha Ke^{\alpha t} + 8 Ke^{\alpha t}) & = 0\\
		 e^{\alpha t} (\alpha^2 + 6\alpha + 8) & = 0\\
		 e^{\alpha t} (\alpha + 4) (\alpha + 2) & = 0\\
		 \alpha_1 = -4, \  \alpha_2 & = -2\\
		 y_h(t) & = c_1 e^{-4 t} + c_2 e^{-2 t}
	\end{split}
	\end{equation}
	Assume the particular solution is of the form $y_p(t) = H(\lambda)e^{\lambda t}$ and we know that $H(\lambda) = \frac{\sum_{0}^{M} b_k \lambda^k}{\sum_{0}^{N} a_k \lambda^k}$; hence, $H(\lambda) = \frac{2}{\lambda^2 + 6\lambda + 8}$. From this, we find the particular solution $y_p(t)$. Notice that, since we are trying to find unit step response $s(t)$, we given the unit step function as the input, i.e.,  $x(t) = u(t)$, equivalently $x(t) = 1$ for $t > 0$.
    	\begin{equation}
	\begin{split}
		\frac{d^2y(t)}{dt^2} + 	6\frac{dy(t)}{dt} + 8y(t) & = 2 u(t)\\
		\frac{2\lambda^2 e^{\lambda t}}{\lambda^2 + 6\lambda + 8} + \frac{12\lambda e^{\lambda t}}{\lambda^2 + 6\lambda + 8} + \frac{16 e^{\lambda t}}{\lambda^2 + 6\lambda + 8}  & = 2 \text{ for } t > 0\\
		\frac{\lambda^2 e^{\lambda t}}{\lambda^2 + 6\lambda + 8} + \frac{6\lambda e^{\lambda t}}{\lambda^2 + 6\lambda + 8} + \frac{8 e^{\lambda t}}{\lambda^2 + 6\lambda + 8}  & = 1 \text{ for } t > 0\\
		\lambda^2 e^{\lambda t} + 6\lambda e^{\lambda t} + 8 e^{\lambda t}  & = \lambda^2 + 6\lambda + 8 \text{ for } t > 0\\
		e^{\lambda t}(\lambda^2  + 6\lambda + 8) & = \lambda^2 + 6\lambda + 8\ \text{ for } t > 0\\
		e^{\lambda t} & = 1 \text{ for } t > 0\\
		\lambda & = 0 \\
		y_p(t) & = \frac{1}{4} 
	\end{split}
	\end{equation}
	Hence, the general solution is given below.
    	\begin{equation}
	\begin{split}
		y(t) & = y_h(t) + y_p(t)\\
		y(t) & = c_1 e^{-4 t} + c_2 e^{-2 t} + \frac{1}{4}
	\end{split}
	\end{equation}
	Since the system is initially at rest, we have $y(0) = 0$ and $y'(0) = 0$. We use them to find $c_1$ and $c_2$.
    	\begin{equation}
	\begin{split}
		y(0) = c_1 + c_2 + \frac{1}{4} & = 0\\
		c_1 + c_2 = -\frac{1}{4}\\
		y'(0) = -4c_1 - 2c_2 & = 0\\
		4c_1 + 2c_2 & = 0\\
		c_1 = \frac{1}{4}, \ c_2 & = -\frac{1}{2}\\
		y(t) = s(t) & = \frac{1}{4} e^{-4 t} - \frac{1}{2} e^{-2 t} + \frac{1}{4}
	\end{split}
	\end{equation}
	Since $\frac{ds(t)}{dt} = h(t)$, we find $h(t)$ as follows.
    	\begin{equation}
	\begin{split}
		h(t) & = \frac{ds(t)}{dt}\\
		h(t) &= (-e^{-4 t} + e^{-2 t})u(t)
	\end{split}
	\end{equation}
	%%%%%%%%%%%%%%%%%%%%%%%%%%%%%%%%%%%%%%%%%%%%%%%%%%%%%%%%%%%%%%%%%%%%%%%%
	%%%%%%%%%%%%%%%%%%%%%%%%%%%%%%%%%%%%%%%%%%%%%%%%%%%%%%%%%%%%%%%%%%%%%%%%
	%%%%%%%%%%%%%%%%%%%%%%%%%%%%%%%%%%%%%%%%%%%%%%%%%%%%%%%%%%%%%%%%%%%%%%%%
	\textbf{An alternative way to solve this problem is as follows.} \\
	This differential equation can be written as,
	\begin{equation}
	\begin{split}
	Q(D)y(t) & = P(D)x(t)  \\
\text{where \quad} Q(D) & = (D^2 + 6D + 8) \\
	P(D) & = 2
	\end{split}
	\end{equation}
	Then the impulse response is,
	\begin{equation}
	h(t) = b_n\delta(t)+(P(D)y_h(t))u(t)
	\end{equation}
	Since the order of $P(D)$ is less than of $Q(D)$, $b_n=0$. Therefore the impulse response can be written as,
	\begin{equation}
	h(t)=(2y_h(t))u(t)
	\end{equation}
	The corresponding homogeneous differential equation is,
	\begin{equation}
	y''(t)+6y'(t)+8y(t)=0 \text{\quad subject to } y_h'(0)=1 \text{, } y_h(0)=0
	\end{equation}
	The characteristic equation of the differential equation above is,
	\begin{equation}
	s^2 + 6s + 8 = 0
	\end{equation}
	Which yields to two different roots $s_1=-2$ and $s_2=-4$. Therefore the homogeneous solution is,
	\begin{equation}
	\begin{split}
	y_h(t) & = Ae^{-2t}+Be^{-4t} \\
	y_h(0) & = A +B = 0 \\
	y_h'(0) &= -2A -4B = 1 \\
	A & = \frac{1}{2}\\
	B & = \frac{-1}{2}\\
	y_h(t) & =  \frac{1}{2}e^{-2t}+\frac{-1}{2}e^{-4t} 
	\end{split}
	\end{equation}
	Therefore the impulse response is,
	\begin{equation}
	h(t) = (e^{-2t}-e^{-4t})u(t)
	\end{equation} \\

    \item 
    \begin{itemize}
		\item \textbf{\textit{Causality:}} The system is causal because $h(t)=0$ for $t<0$. $u(t)$ makes it causal.\\
		\item \textbf{\textit{Memory:}} The system is not memoryless since $h(t)$ cannot be written in the form of $K\delta(t)$ where $K$ is a constant. In other words, the system has memory. \\ 
		\item \textbf{\textit{Stability:}} If the system is stable then,
		\begin{equation}
		\begin{split}
		&\int_{-\infty}^{\infty}|h(\tau)|d\tau < \infty \\
		-\infty < & \int_{-\infty}^{\infty}h(\tau)d\tau < \infty
		\end{split}
		\end{equation}
		We plug in the impulse response, 
		\begin{equation}
		\begin{split}
		\int_{-\infty}^{\infty}h(\tau)d\tau = &\int_{-\infty}^{\infty}(e^{-2\tau}-e^{-4\tau})u(\tau)d\tau \\
		= &\int_{0}^{\infty}(e^{-2\tau}-e^{-4\tau})d\tau \\
		= &\lim_{R\rightarrow \infty}\int_{0}^{R}(e^{-2\tau}-e^{-4\tau})d\tau \\
		= &\lim_{R\rightarrow \infty}(\frac{e^{-4R}}{4}-\frac{e^{-2R}}{2}-\frac{1}{4}+\frac{1}{2}) \\
		= & - \frac{1}{4} + \frac{1}{2} = \frac{1}{4}
		\end{split}
		\end{equation}	 
		Since $\frac{1}{4}$ is finite, the system is stable. \\
		\item \textbf{\textit{Invertibility:}} The system is not invertible since there no system $h_1(t)$ that makes $h(t)* h_1(t)=\delta(t)$.
	\end{itemize}
    \end{enumerate}

\end{enumerate}
\end{document}

