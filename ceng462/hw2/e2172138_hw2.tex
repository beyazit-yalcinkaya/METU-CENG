\documentclass[12pt]{article}
\usepackage[utf8]{inputenc}
\usepackage{float}
\usepackage{amsmath}
\usepackage{caption}


\usepackage[hmargin=3cm,vmargin=6.0cm]{geometry}
%\topmargin=0cm
\topmargin=-2cm
\addtolength{\textheight}{6.5cm}
\addtolength{\textwidth}{2.0cm}
%\setlength{\leftmargin}{-5cm}
\setlength{\oddsidemargin}{0.0cm}
\setlength{\evensidemargin}{0.0cm}

\newcommand{\RN}[1]{%
  \textup{\uppercase\expandafter{\romannumeral#1}}%
}

\title{CENG 462 - Artificial Intelligence Homework 2}
\author{Beyazıt Yalçınkaya\\2172138}
\date{}



\begin{document}
\maketitle

\section*{\RN{1}.}
	\begin{itemize}
		\item[\textbf{a.}] Below, given problem is defined as a constraint satisfaction problem. Notice that a constraint satisfaction problem is formally defined as a triple $(X, D, C)$, where $X$ is a set of variables, $D$ is a set of their respective domains of values, and $C$ is a set of constraints.
		
		Given scheduling problem is defined as a triple $(X, D, C)$, where
			\begin{itemize}
				\item
					\begin{align*}
					X = \{\texttt{CENG111}, \texttt{CENG213}, \texttt{CENG223}, \texttt{CENG315}, \texttt{CENG331}, \texttt{CENG351}\}
					\end{align*}
				\item
					\begin{align*}
					D = \{(x, \delta) \mid \forall x \in X \}\text{, where }\delta = \{&(\texttt{BMB1, 09:30}), (\texttt{BMB1, 13:30}),\\
						&(\texttt{BMB2, 09:30}), (\texttt{BMB2, 13:30}),\\
						&(\texttt{BMB3, 13:30}), (\texttt{BMB3, 16:30})\}
					\end{align*}
				\item
					\begin{align*}
					C = \big\{&\texttt{TIME[CENG213]} \neq \texttt{TIME[CENG223]}, \texttt{TIME[CENG315]} \neq \texttt{TIME[CENG331]},\\
						&\texttt{TIME[CENG331]} \neq \texttt{TIME[CENG351]}, \texttt{TIME[CENG315]} \neq \texttt{TIME[CENG351]},\\
						&\forall x \forall y \big(x \in X \land y \in X \land x \neq y \land (\texttt{CLASS[$x$]}, \texttt{TIME[$x$]}) \neq (\texttt{CLASS[$y$]}, \texttt{TIME[$y$]})\big)\big\}
					\end{align*}
			\end{itemize}


		\item[\textbf{b.}] Table for the application of backtracking by forward checking is given below.

\begin{table}[H]
\centering
\label{my-label1}
\begin{tabular}{|c|c|c|c|c|c|}
\hline
\texttt{CENG111} & \texttt{CENG213} & \texttt{CENG223} & \texttt{CENG315} & \texttt{CENG331} & \texttt{CENG351}\\ \hline
\tiny\shortstack{\{(\texttt{BMB1, 09:30}),\\ (\texttt{BMB1, 13:30}),\\ (\texttt{BMB2, 09:30}),\\ (\texttt{BMB2, 13:30}),\\ (\texttt{BMB3, 13:30}),\\ (\texttt{BMB3, 16:30})\}} & \tiny\shortstack{\{(\texttt{BMB1, 09:30}),\\ (\texttt{BMB1, 13:30}),\\ (\texttt{BMB2, 09:30}),\\ (\texttt{BMB2, 13:30}),\\ (\texttt{BMB3, 13:30}),\\ (\texttt{BMB3, 16:30})\}} & \tiny\shortstack{\{(\texttt{BMB1, 09:30}),\\ (\texttt{BMB1, 13:30}),\\ (\texttt{BMB2, 09:30}),\\ (\texttt{BMB2, 13:30}),\\ (\texttt{BMB3, 13:30}),\\ (\texttt{BMB3, 16:30})\}} & \tiny\shortstack{\{(\texttt{BMB1, 09:30}),\\ (\texttt{BMB1, 13:30}),\\ (\texttt{BMB2, 09:30}),\\ (\texttt{BMB2, 13:30}),\\ (\texttt{BMB3, 13:30}),\\ (\texttt{BMB3, 16:30})\}} & \tiny\shortstack{\{(\texttt{BMB1, 09:30}),\\ (\texttt{BMB1, 13:30}),\\ (\texttt{BMB2, 09:30}),\\ (\texttt{BMB2, 13:30}),\\ (\texttt{BMB3, 13:30}),\\ (\texttt{BMB3, 16:30})\}} & \tiny\shortstack{\{(\texttt{BMB1, 09:30}),\\ (\texttt{BMB1, 13:30}),\\ (\texttt{BMB2, 09:30}),\\ (\texttt{BMB2, 13:30}),\\ (\texttt{BMB3, 13:30}),\\ (\texttt{BMB3, 16:30})\}}\\ \hline
(\texttt{BMB3, 16:30}) & \tiny\shortstack{\{(\texttt{BMB1, 09:30}),\\ (\texttt{BMB1, 13:30}),\\ (\texttt{BMB2, 09:30}),\\ (\texttt{BMB2, 13:30}),\\ (\texttt{BMB3, 13:30})\}} & \tiny\shortstack{\{(\texttt{BMB1, 09:30}),\\ (\texttt{BMB1, 13:30}),\\ (\texttt{BMB2, 09:30}),\\ (\texttt{BMB2, 13:30}),\\ (\texttt{BMB3, 13:30})\}} & \tiny\shortstack{\{(\texttt{BMB1, 09:30}),\\ (\texttt{BMB1, 13:30}),\\ (\texttt{BMB2, 09:30}),\\ (\texttt{BMB2, 13:30}),\\ (\texttt{BMB3, 13:30})\}} & \tiny\shortstack{\{(\texttt{BMB1, 09:30}),\\ (\texttt{BMB1, 13:30}),\\ (\texttt{BMB2, 09:30}),\\ (\texttt{BMB2, 13:30}),\\ (\texttt{BMB3, 13:30})\}} & \tiny\shortstack{\{(\texttt{BMB1, 09:30}),\\ (\texttt{BMB1, 13:30}),\\ (\texttt{BMB2, 09:30}),\\ (\texttt{BMB2, 13:30}),\\ (\texttt{BMB3, 13:30})\}}\\ \hline
(\texttt{BMB3, 16:30}) & \tiny\shortstack{\{(\texttt{BMB1, 13:30}),\\ (\texttt{BMB2, 09:30}),\\ (\texttt{BMB2, 13:30}),\\ (\texttt{BMB3, 13:30})\}} & \tiny\shortstack{\{(\texttt{BMB1, 13:30}),\\ (\texttt{BMB2, 09:30}),\\ (\texttt{BMB2, 13:30}),\\ (\texttt{BMB3, 13:30})\}} & (\texttt{BMB1, 09:30}) & \tiny\shortstack{\{(\texttt{BMB1, 13:30}),\\ (\texttt{BMB2, 13:30}),\\ (\texttt{BMB3, 13:30})\}} & \tiny\shortstack{\{(\texttt{BMB1, 13:30}),\\ (\texttt{BMB2, 13:30}),\\ (\texttt{BMB3, 13:30})\}}\\ \hline
(\texttt{BMB3, 16:30}) & \tiny\shortstack{\{(\texttt{BMB2, 09:30}),\\ (\texttt{BMB2, 13:30}),\\ (\texttt{BMB3, 13:30})\}} & \tiny\shortstack{\{(\texttt{BMB2, 09:30}),\\ (\texttt{BMB2, 13:30}),\\ (\texttt{BMB3, 13:30})\}} & (\texttt{BMB1, 09:30}) & (\texttt{BMB1, 13:30}) & $\emptyset$\\ \hline
\end{tabular}
\end{table} 




		\item[\textbf{c.}] Table for the application of backtracking by arc consistency is given below.

		
\begin{table}[H]
\centering
\label{my-label1}
\begin{tabular}{|c|c|c|c|c|c|}
\hline
\texttt{CENG111} & \texttt{CENG213} & \texttt{CENG223} & \texttt{CENG315} & \texttt{CENG331} & \texttt{CENG351}\\ \hline
\tiny\shortstack{\{(\texttt{BMB1, 09:30}),\\ (\texttt{BMB1, 13:30}),\\ (\texttt{BMB2, 09:30}),\\ (\texttt{BMB2, 13:30}),\\ (\texttt{BMB3, 13:30}),\\ (\texttt{BMB3, 16:30})\}} & \tiny\shortstack{\{(\texttt{BMB1, 09:30}),\\ (\texttt{BMB1, 13:30}),\\ (\texttt{BMB2, 09:30}),\\ (\texttt{BMB2, 13:30}),\\ (\texttt{BMB3, 13:30}),\\ (\texttt{BMB3, 16:30})\}} & \tiny\shortstack{\{(\texttt{BMB1, 09:30}),\\ (\texttt{BMB1, 13:30}),\\ (\texttt{BMB2, 09:30}),\\ (\texttt{BMB2, 13:30}),\\ (\texttt{BMB3, 13:30}),\\ (\texttt{BMB3, 16:30})\}} & \tiny\shortstack{\{(\texttt{BMB1, 09:30}),\\ (\texttt{BMB1, 13:30}),\\ (\texttt{BMB2, 09:30}),\\ (\texttt{BMB2, 13:30}),\\ (\texttt{BMB3, 13:30}),\\ (\texttt{BMB3, 16:30})\}} & \tiny\shortstack{\{(\texttt{BMB1, 09:30}),\\ (\texttt{BMB1, 13:30}),\\ (\texttt{BMB2, 09:30}),\\ (\texttt{BMB2, 13:30}),\\ (\texttt{BMB3, 13:30}),\\ (\texttt{BMB3, 16:30})\}} & \tiny\shortstack{\{(\texttt{BMB1, 09:30}),\\ (\texttt{BMB1, 13:30}),\\ (\texttt{BMB2, 09:30}),\\ (\texttt{BMB2, 13:30}),\\ (\texttt{BMB3, 13:30}),\\ (\texttt{BMB3, 16:30})\}}\\ \hline
(\texttt{BMB3, 16:30}) & \tiny\shortstack{\{(\texttt{BMB1, 09:30}),\\ (\texttt{BMB1, 13:30}),\\ (\texttt{BMB2, 09:30}),\\ (\texttt{BMB2, 13:30}),\\ (\texttt{BMB3, 13:30})\}} & \tiny\shortstack{\{(\texttt{BMB1, 09:30}),\\ (\texttt{BMB1, 13:30}),\\ (\texttt{BMB2, 09:30}),\\ (\texttt{BMB2, 13:30}),\\ (\texttt{BMB3, 13:30})\}} & \tiny\shortstack{\{(\texttt{BMB1, 09:30}),\\ (\texttt{BMB1, 13:30}),\\ (\texttt{BMB2, 09:30}),\\ (\texttt{BMB2, 13:30}),\\ (\texttt{BMB3, 13:30})\}} & \tiny\shortstack{\{(\texttt{BMB1, 09:30}),\\ (\texttt{BMB1, 13:30}),\\ (\texttt{BMB2, 09:30}),\\ (\texttt{BMB2, 13:30}),\\ (\texttt{BMB3, 13:30})\}} & \tiny\shortstack{\{(\texttt{BMB1, 09:30}),\\ (\texttt{BMB1, 13:30}),\\ (\texttt{BMB2, 09:30}),\\ (\texttt{BMB2, 13:30}),\\ (\texttt{BMB3, 13:30})\}}\\ \hline
(\texttt{BMB3, 16:30}) & \tiny\shortstack{\{(\texttt{BMB1, 13:30}),\\ (\texttt{BMB2, 09:30}),\\ (\texttt{BMB2, 13:30}),\\ (\texttt{BMB3, 13:30})\}} & \tiny\shortstack{\{(\texttt{BMB1, 13:30}),\\ (\texttt{BMB2, 09:30}),\\ (\texttt{BMB2, 13:30}),\\ (\texttt{BMB3, 13:30})\}} & (\texttt{BMB1, 09:30}) & \tiny\shortstack{\{(\texttt{BMB1, 13:30}),\\ (\texttt{BMB2, 13:30}),\\ (\texttt{BMB3, 13:30})\}} & $\emptyset$\\ \hline
\end{tabular}
\end{table}

	\end{itemize}


\section*{\RN{2}.}
\begin{table}[H]
\centering
\label{my-label1}
\begin{tabular}{|c|c|c|c|}
\hline
node & v & $\alpha$ & $\beta$\\ \hline
A & $5$ & $5$ & $+\infty$\\ \hline
B & $5$ & $-\infty$ & $5$\\ \hline
C & $2$  & $5$ & $2$\\ \hline
D & $5$ & $5$ & $+\infty$\\ \hline
E & $6$ & $6$ & $5$\\ \hline
F & $2$ & $5$ & $+\infty$\\ \hline
G & $-$ & $-$ & $-$\\ \hline
\end{tabular}
\end{table}

Notice that the node G and the final states with value 0, 1, and 10 are not explored, i.e., they are pruned by the $\alpha/\beta$ pruning.



\section*{\RN{3}.}



	\begin{itemize}
		\item[\textbf{a.}] Below, a sequence of tables are given for the forward chaining.
		
\begin{table}[H]
\centering
\caption*{Step 1}
\label{my-label1}
\begin{tabular}{ c c }
\hline
$c$ & count[$c$]\\ \hline
$K \implies L$ & 1\\
$I \land J \implies K$ & 2\\
$G \land H \implies I$ & 2\\
$H \land D \implies J$ & 2\\
$E \land H \implies G$ & 2\\
$E \land F \implies H$ & 2\\
$G \land A \implies E$ & 2\\
$A \land B \implies E$ & 2\\
$B \land C \implies F$ & 2\\
$A$ & 0\\
$B$ & 0\\
$C$ & 0\\
$D$ & 0\\ \hline
$agenda$ & $(A, B, C, D)$
\end{tabular}
\end{table}
		
\begin{table}[H]
\centering
\caption*{Step 2}
\label{my-label1}
\begin{tabular}{ c c }
\hline
$c$ & count[$c$]\\ \hline
$K \implies L$ & 1\\
$I \land J \implies K$ & 2\\
$G \land H \implies I$ & 2\\
$H \land D \implies J$ & 2\\
$E \land H \implies G$ & 2\\
$E \land F \implies H$ & 2\\
$G \land A \implies E$ & 1\\
$A \land B \implies E$ & 1\\
$B \land C \implies F$ & 2\\
$A$ & 0\\
$B$ & 0\\
$C$ & 0\\
$D$ & 0\\ \hline
$agenda$ & $(B, C, D)$
\end{tabular}
\end{table}
		
\begin{table}[H]
\centering
\caption*{Step 3}
\label{my-label1}
\begin{tabular}{ c c }
\hline
$c$ & count[$c$]\\ \hline
$K \implies L$ & 1\\
$I \land J \implies K$ & 2\\
$G \land H \implies I$ & 2\\
$H \land D \implies J$ & 2\\
$E \land H \implies G$ & 2\\
$E \land F \implies H$ & 2\\
$G \land A \implies E$ & 1\\
$A \land B \implies E$ & 0\\
$B \land C \implies F$ & 1\\
$A$ & 0\\
$B$ & 0\\
$C$ & 0\\
$D$ & 0\\ \hline
$agenda$ & $(C, D, E)$
\end{tabular}
\end{table}
		
\begin{table}[H]
\centering
\caption*{Step 4}
\label{my-label1}
\begin{tabular}{ c c }
\hline
$c$ & count[$c$]\\ \hline
$K \implies L$ & 1\\
$I \land J \implies K$ & 2\\
$G \land H \implies I$ & 2\\
$H \land D \implies J$ & 2\\
$E \land H \implies G$ & 2\\
$E \land F \implies H$ & 2\\
$G \land A \implies E$ & 1\\
$A \land B \implies E$ & 0\\
$B \land C \implies F$ & 0\\
$A$ & 0\\
$B$ & 0\\
$C$ & 0\\
$D$ & 0\\ \hline
$agenda$ & $(D, E, F)$
\end{tabular}
\end{table}
		
\begin{table}[H]
\centering
\caption*{Step 5}
\label{my-label1}
\begin{tabular}{ c c }
\hline
$c$ & count[$c$]\\ \hline
$K \implies L$ & 1\\
$I \land J \implies K$ & 2\\
$G \land H \implies I$ & 2\\
$H \land D \implies J$ & 1\\
$E \land H \implies G$ & 2\\
$E \land F \implies H$ & 2\\
$G \land A \implies E$ & 1\\
$A \land B \implies E$ & 0\\
$B \land C \implies F$ & 0\\
$A$ & 0\\
$B$ & 0\\
$C$ & 0\\
$D$ & 0\\ \hline
$agenda$ & $(E, F)$
\end{tabular}
\end{table}
		
\begin{table}[H]
\centering
\caption*{Step 6}
\label{my-label1}
\begin{tabular}{ c c }
\hline
$c$ & count[$c$]\\ \hline
$K \implies L$ & 1\\
$I \land J \implies K$ & 2\\
$G \land H \implies I$ & 2\\
$H \land D \implies J$ & 1\\
$E \land H \implies G$ & 1\\
$E \land F \implies H$ & 1\\
$G \land A \implies E$ & 1\\
$A \land B \implies E$ & 0\\
$B \land C \implies F$ & 0\\
$A$ & 0\\
$B$ & 0\\
$C$ & 0\\
$D$ & 0\\ \hline
$agenda$ & $(F)$
\end{tabular}
\end{table}
		
\begin{table}[H]
\centering
\caption*{Step 7}
\label{my-label1}
\begin{tabular}{ c c }
\hline
$c$ & count[$c$]\\ \hline
$K \implies L$ & 1\\
$I \land J \implies K$ & 2\\
$G \land H \implies I$ & 2\\
$H \land D \implies J$ & 1\\
$E \land H \implies G$ & 1\\
$E \land F \implies H$ & 0\\
$G \land A \implies E$ & 1\\
$A \land B \implies E$ & 0\\
$B \land C \implies F$ & 0\\
$A$ & 0\\
$B$ & 0\\
$C$ & 0\\
$D$ & 0\\ \hline
$agenda$ & $(H)$
\end{tabular}
\end{table}
		
\begin{table}[H]
\centering
\caption*{Step 8}
\label{my-label1}
\begin{tabular}{ c c }
\hline
$c$ & count[$c$]\\ \hline
$K \implies L$ & 1\\
$I \land J \implies K$ & 2\\
$G \land H \implies I$ & 1\\
$H \land D \implies J$ & 0\\
$E \land H \implies G$ & 0\\
$E \land F \implies H$ & 0\\
$G \land A \implies E$ & 1\\
$A \land B \implies E$ & 0\\
$B \land C \implies F$ & 0\\
$A$ & 0\\
$B$ & 0\\
$C$ & 0\\
$D$ & 0\\ \hline
$agenda$ & $(G, J)$
\end{tabular}
\end{table}
		
\begin{table}[H]
\centering
\caption*{Step 9}
\label{my-label1}
\begin{tabular}{ c c }
\hline
$c$ & count[$c$]\\ \hline
$K \implies L$ & 1\\
$I \land J \implies K$ & 2\\
$G \land H \implies I$ & 0\\
$H \land D \implies J$ & 0\\
$E \land H \implies G$ & 0\\
$E \land F \implies H$ & 0\\
$G \land A \implies E$ & 0\\
$A \land B \implies E$ & 0\\
$B \land C \implies F$ & 0\\
$A$ & 0\\
$B$ & 0\\
$C$ & 0\\
$D$ & 0\\ \hline
$agenda$ & $(J, I)$
\end{tabular}
\end{table}
		
\begin{table}[H]
\centering
\caption*{Step 10}
\label{my-label1}
\begin{tabular}{ c c }
\hline
$c$ & count[$c$]\\ \hline
$K \implies L$ & 1\\
$I \land J \implies K$ & 1\\
$G \land H \implies I$ & 0\\
$H \land D \implies J$ & 0\\
$E \land H \implies G$ & 0\\
$E \land F \implies H$ & 0\\
$G \land A \implies E$ & 0\\
$A \land B \implies E$ & 0\\
$B \land C \implies F$ & 0\\
$A$ & 0\\
$B$ & 0\\
$C$ & 0\\
$D$ & 0\\ \hline
$agenda$ & $(I)$
\end{tabular}
\end{table}
		
\begin{table}[H]
\centering
\caption*{Step 11}
\label{my-label1}
\begin{tabular}{ c c }
\hline
$c$ & count[$c$]\\ \hline
$K \implies L$ & 1\\
$I \land J \implies K$ & 0\\
$G \land H \implies I$ & 0\\
$H \land D \implies J$ & 0\\
$E \land H \implies G$ & 0\\
$E \land F \implies H$ & 0\\
$G \land A \implies E$ & 0\\
$A \land B \implies E$ & 0\\
$B \land C \implies F$ & 0\\
$A$ & 0\\
$B$ & 0\\
$C$ & 0\\
$D$ & 0\\ \hline
$agenda$ & $(K)$
\end{tabular}
\end{table}
		
\begin{table}[H]
\centering
\caption*{Step 12}
\label{my-label1}
\begin{tabular}{ c c }
\hline
$c$ & count[$c$]\\ \hline
$K \implies L$ & 0\\
$I \land J \implies K$ & 0\\
$G \land H \implies I$ & 0\\
$H \land D \implies J$ & 0\\
$E \land H \implies G$ & 0\\
$E \land F \implies H$ & 0\\
$G \land A \implies E$ & 0\\
$A \land B \implies E$ & 0\\
$B \land C \implies F$ & 0\\
$A$ & 0\\
$B$ & 0\\
$C$ & 0\\
$D$ & 0\\ \hline
$agenda$ & $(L)$
\end{tabular}
\end{table}
		
\begin{table}[H]
\centering
\caption*{Final, $K \implies L$ is proved.}
\label{my-label1}
\begin{tabular}{ c c }
\hline
$c$ & count[$c$]\\ \hline
$K \implies L$ & 0\\
$I \land J \implies K$ & 0\\
$G \land H \implies I$ & 0\\
$H \land D \implies J$ & 0\\
$E \land H \implies G$ & 0\\
$E \land F \implies H$ & 0\\
$G \land A \implies E$ & 0\\
$A \land B \implies E$ & 0\\
$B \land C \implies F$ & 0\\
$A$ & 0\\
$B$ & 0\\
$C$ & 0\\
$D$ & 0\\ \hline
$agenda$ & $()$
\end{tabular}
\end{table}

\item[\textbf{b.}] Below, a sequence of steps are given for the backward chaining.
\begin{enumerate}
	\item Start with the goal $K \implies L$. Find proof for $K$.
	\item For $K$, $I$ and $J$ must be proved.
	\item For $I$, $G$ and $H$ must be proved.
	\item For $J$, $H$ and $D$ must be proved.
	\item For $G$, $E$ and $H$ must be proved.
	\item For $H$, $E$ and $F$ must be proved.
	\item For $E$, $A$ and $B$ must be proved. Since both are in the knowledge base, $E$ is proved and added to the knowledge base.
	\item For $F$, $B$ and $C$ must be proved. Since both are in the knowledge base, $F$ is proved and added to the knowledge base.
	\item Using $E$ and $F$, $H$ is proved and added to the knowledge base.
	\item Using $E$ and $H$, $G$ is proved and added to the knowledge base.
	\item Using $H$ and $D$, $J$ is proved and added to the knowledge base.
	\item Using $G$ and $H$, $I$ is proved and added to the knowledge base.
	\item Using $I$ and $J$, $K$ is proved and added to the knowledge base.
	\item Using $K$, $L$ is proved; hence, the goal proposition $K \implies L$ is proved.
\end{enumerate}


	\end{itemize}

\end{document}

​
