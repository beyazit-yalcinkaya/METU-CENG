\documentclass[12pt]{article}
\usepackage[utf8]{inputenc}
\usepackage{float}
\usepackage{amsmath}
\usepackage{enumitem}
\usepackage[justification=centering]{caption}

\usepackage[hmargin=3cm,vmargin=6.0cm]{geometry}
%\topmargin=0cm
\topmargin=-2cm
\addtolength{\textheight}{6.5cm}
\addtolength{\textwidth}{2.0cm}
%\setlength{\leftmargin}{-5cm}
\setlength{\oddsidemargin}{0.0cm}
\setlength{\evensidemargin}{0.0cm}

%misc libraries goes here
\usepackage{tikz}
\usetikzlibrary{automata, positioning}
\newtheorem{defn}{Definition}

\author{Beyazit Yalcinkaya\\2172138}
\title{\textbf{CENG 798 QUANTUM COMPUTING\\ASSIGNMENT}}
\date{}
\begin{document}
\maketitle

%Write your full name and id number between the colon and newline
%Put one empty space character after colon and before newline


% Write your answers below the section tags
\section*{Answer 1}

\subsection*{a.} 

Below, $|\psi\rangle$ is given in the canonical basis, i.e., $\{|0\rangle, |1\rangle\}$.

\begin{equation}
\begin{split}
|\psi\rangle & = \cos\left(\frac{\pi}{3}\right)|0\rangle + \sin\left(\frac{\pi}{3}\right)|1\rangle\\
& = \frac{1}{2}|0\rangle + \frac{\sqrt{3}}{2}|1\rangle
\end{split}
\end{equation}

\subsection*{b.}

$|\phi\rangle$ is found as follows.

\begin{equation}
\begin{split}
|\phi\rangle & = NOT \ |\psi\rangle = \begin{bmatrix}$$0$$ & $$1$$\\$$1$$ & $$0$$\end{bmatrix}\begin{bmatrix}$$1/2$$\\$$\sqrt{3}/2$$\end{bmatrix} = \begin{bmatrix}$$\sqrt{3}/2$$\\$$1/2$$\end{bmatrix}\\
& = \frac{\sqrt{3}}{2}|0\rangle + \frac{1}{2}|1\rangle
\end{split}
\end{equation}
Then, we find $|\psi\rangle \otimes |\phi\rangle$ below.

\begin{equation}
\begin{split}
|\psi\rangle \otimes |\phi\rangle & = \begin{bmatrix}$$1/2$$\\$$\sqrt{3}/2$$\end{bmatrix} \otimes \begin{bmatrix}$$\sqrt{3}/2$$\\$$1/2$$\end{bmatrix}  = \begin{bmatrix}$$\sqrt{3}/4$$\\$$1/4$$\\$$3/4$$\\$$\sqrt{3}/4$$\end{bmatrix}\\
& = \frac{\sqrt{3}}{4}|00\rangle + \frac{1}{4}|01\rangle + \frac{3}{4}|10\rangle + \frac{\sqrt{3}}{4}|11\rangle
\end{split}
\end{equation}


\section*{Answer 2} 

\subsection*{a.} 

First, we write $|0\rangle$ and $|1\rangle$ in the $\{|u\rangle, |w\rangle\}$ basis.

\begin{equation}\label{zero}
|0\rangle = \frac{1}{\sqrt{2}}|u\rangle + \frac{1}{\sqrt{2}}|w\rangle
\end{equation}

\begin{equation}\label{one}
|1\rangle = \frac{1}{\sqrt{2}}|u\rangle - \frac{1}{\sqrt{2}}|w\rangle
\end{equation}
Using Eqn.~\ref{zero} and Eqn.~\ref{one}, we write $|\psi\rangle$ in the $\{|u\rangle, |w\rangle\}$ basis.

\begin{equation}
\begin{split}
|\psi\rangle & = \frac{1}{2}|0\rangle + \frac{\sqrt{3}}{2}|1\rangle \\
& = \frac{1}{2}\left( \frac{1}{\sqrt{2}}|u\rangle + \frac{1}{\sqrt{2}}|w\rangle \right) + \frac{\sqrt{3}}{2}\left( \frac{1}{\sqrt{2}}|u\rangle - \frac{1}{\sqrt{2}}|w\rangle \right)\\
& = \frac{\left( 1+\sqrt{3} \right)}{2\sqrt{2}}|u\rangle + \frac{\left(1-\sqrt{3}\right)}{2\sqrt{2}}|w\rangle
\end{split}
\end{equation}

\subsection*{b.}

The probability of observing $|u\rangle$ as the new state after a measurement is given below.

\begin{equation}
\left( \frac{\left( 1+\sqrt{3} \right)}{2\sqrt{2}} \right)^{2} = \frac{\left( 2+\sqrt{3} \right)}{4}
\end{equation}

\end{document}

​